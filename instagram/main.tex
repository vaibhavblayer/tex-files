\documentclass{article}
\usepackage[utf8]{inputenc}
\usepackage{v-laws-of-motion}
\geometry{paperheight=5 in, paperwidth=5 in, top=10mm, bottom=15mm, left=10mm, right=15mm}

\usepackage[utopia]{mathdesign}
\pagenumbering{gobble}
%\newcommand{\ans}{\textcolor{red!95}{\textit{\quad}}}
%\def\ansit#1{\textcolor{red!95}{\quad}}
\newcommand{\ans}{\textcolor{red!95}{\textit{\quad Ans.}}}
\def\ansit#1{\textcolor{red!95}{\quad [ #1 ]}}

\title{Instagram(Laws of Motionf)\\(Problems)\\01}

\tikzstyle test-paper=[>=latex, thick]
\tikzset{
>=latex
}

\def\surface#1{
	\fill[pattern=north east lines](0, 0)--(8, 0)--(8, -3)--(7.75, -3)--(7.75, -0.25)--(0, -0.25)--cycle;
	\draw[thick](0, 0)--(8, 0)--(8, -3);
	}

\tikzstyle{block}=[rectangle,draw, thick, minimum size=8mm, node distance=1.6cm]
\tikzstyle{pulley}=[circle,draw, thick, minimum size=10mm, node distance=2cm]

\tikzstyle{sblock}=[rectangle,draw, thick, minimum size=8mm, node distance=1.8cm]
\tikzstyle{spulley}=[circle,draw, thick, minimum size=8mm, node distance=1.6cm]

\tikzstyle{lblock}=[rectangle,draw, thick, minimum size=12mm, node distance=1.8cm]
\tikzstyle{lpulley}=[circle,draw, thick, minimum size=12mm, node distance=2cm]

\tikzstyle{hblock}=[rectangle,draw, thick, minimum height=12mm, minimum width=20mm, node distance=1.8cm]
\tikzstyle{Hblock}=[rectangle,draw, thick, minimum height=12mm, minimum width=24mm, node distance=1.8cm]
\tikzstyle{lift}=[rectangle,draw, thick, minimum height=60mm, minimum width=50mm, node distance=1.8cm]

\tikzstyle{Bpulley}=[circle,draw, thick, minimum size=20mm, node distance=1.8cm]

\tikzstyle{plank}=[rectangle,draw, thick, minimum height=8mm, minimum width=50mm, node distance=1.8cm]



\begin{document}
\maketitle

\pagebreak

\begin{center}
\texttt{Laws of Motion}
\end{center}

\begin{enumerate}
\item[01.] Hello this a problem from laws of motion. Three blocks are connected through strings and pulleys as shown in figure. Find the acceleration of the system.

\begin{center}
\begin{tikzpicture}
\def\ph{0.75}%pulley-height
\def\tl{6}%table-length
	\fill[pattern=north east lines](0, -3)--(0, 0)--(\tl, 0)--(\tl, -3)--(\tl - 0.25, -3)--(\tl - 0.25, -0.25)--(0.25, -0.25)--(0.25, -3)--(0, -3)--cycle;
	\draw[thick](0, -3)--(0, 0)--(\tl, 0)--(\tl, -3);
	\node[pulley] (pulley1) at (-\ph, 0){};
	\node[pulley] (pulley2) at (\tl+\ph, 0){};
	\tzdot*(pulley1.center)
	\node[block, scale=1.25] (block1) at (-\ph-0.5, -2){$15\kg$};
	\node[block, scale=1.25] (block3) at (0.5*\tl, 0.5){$5\kg$};
	\tzline(pulley1.west)(block1.north)
	\tzline[-->--](block3.west)(pulley1.north)
	\tzline[-->--](block3.east)(pulley2.north)
	
	\tzline(pulley2.center)(\tl, 0)
	\tzline(pulley1.center)(0, 0)
	\tzdot*(pulley2.center)
	\node[block, scale=1.25] (block2) at (\tl+\ph+0.5, -2){$5\kg$};
	\tzline(pulley2.east)(block2.north)
\end{tikzpicture}
\end{center}
\begin{tasks}(2)
	\task $9.2$
	\task $7.8$\ans
	\task $4$
	\task $9.8$
\end{tasks}
\end{enumerate}

\pagebreak

\begin{center}
\begin{tikzpicture}
\pic (surface) {frame=9cm};
	\node[block, yshift=4mm] (block1) at (surface-center){$B$};
	\node[block] (block2) [right of=block1]{$A$};
	\node[block] (block3) [left of=block1]{$C$};
	\tzline[-->--=0.5](block3.east)(block1.west)
	\tzline[-->--=0.5](block1.east)(block2.west)
	\tzline+[->](block2.east)(1.5, 0){$F$}[r]
\end{tikzpicture}
\end{center}	


\begin{center}
\begin{tikzpicture}
	\pic[xshift=1cm] (0, 0) {frame=8cm};
	\tzcoors(0, 0)(A)(3, 0)(B)(0,3)(C);
	\tzpolygon(A)(B)(C);
	\tzanglemark(C)(B)(A){$45^\circ$}(15pt)
	\node[block, yshift=4mm, rotate around={-45:(B)}] (box1) at (0, 0) {$10\kg$};
	\tzline+[->] (box1.east)(1*cos{45}, -1*sin{45}){$3\N$}[b]
	\tzline+[->] (box1.west)(-1*cos{45}, 1*sin{45}){$F$}[a]
\end{tikzpicture}
\end{center}
\begin{tasks}(2)
	\task $32\N$\ans
	\task $25\N$
	\task $23\N$
	\task $18\N$
\end{tasks}







\end{document}
