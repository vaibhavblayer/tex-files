\documentclass{article}
\usepackage{v-test-paper}
%\renewcommand{\ans}{\quad}
%\def\ansint#1{\quad}
\title{Module-Test-11\\(Physics-JEE)}

%	TOPICS
% -> Work Done
% -> Potential Energy
% -> Kinetic Energy
% -> Power : Average : Instantaneous

\begin{document}

\maketitle

\jeeSectionA
\begin{enumerate}
\item The total work done on a particle is equal to the change in its kinetic energy
\begin{tasks}(1)
	\task always\ans
	\task only if the forces acting on the body are conservative
	\task only in the inertial frame
	\task only if no external force is acting
\end{tasks}

\item Identify, which of the following energies can be positive (or zero) only?
\begin{tasks}(1)
	\task Kinetic energy\ans
	\task Potential energy
	\task Mechanical energy
	\task Both kinetic and mechanical energies
\end{tasks}

\item A pump is required to lift $800 \kg$ of water per minute from a $10 \m$ deep well and eject it with speed of $20 \mps$. The required power in watts of the pump will be
\begin{tasks}(2)
	\task $6000$
	\task $4000$\ans
	\task $5000$
	\task $8000$
\end{tasks}

\item A particle of mass $m$ moves from rest under the action of a constant force $F$ which acts for two seconds. The maximum power attained is
\begin{tasks}(2)
	\task $2Fm$
	\task $\dfrac{F^2}{m}$
	\task $\dfrac{2F}{m}$
	\task $\dfrac{2F^2}{m}$\ans
\end{tasks}

\item A bullet moving with a speed of $100 \mps$ can just penetrate into two planks of equal thickness. Then the number of such planks, if speed is doubled will be
\begin{tasks}(2)
	\task $6$
	\task $10$
	\task $4$
	\task $8$\ans
\end{tasks}

\item Power applied to a particle varies with time as $P = (3t^2 - 2t + 1) \Watt$, where $t$ is in second. Find the change in its kinetic energy between time $t = 2 \s$ and $t = 4 \s$
\begin{tasks}(2)
	\task $32\Joule$
	\task $46\Joule$\ans
	\task $61\Joule$
	\task $102\Joule$
\end{tasks}

\item A block of mass $10 \kg$ is moving in x-direction with a constant speed of $10 \mps$. It is subjected to a retarding force $F = - 0.1 x \Joule/\m$ during its travel from $x = 20 \m$ to $x = 30 \m$. Its final kinetic energy will be
\begin{tasks}(2)
	\task $475\Joule$\ans
	\task $450\Joule$
	\task $275\Joule$
	\task $250\Joule$
\end{tasks}

\item The kinetic energy of a projectile at its highest position is $K$. If the range of the projectile is four times the height of the projectile($R=4H$), then the initial kinetic energy of the projectile is
\begin{tasks}(2)
	\task $\sqrt{2}K$
	\task $2K$\ans
	\task $4K$
	\task $2\sqrt{2}K$
\end{tasks}

\item 

\end{enumerate}

\jeeSectionB
\begin{enumerate}\addtocounter{enumi}{20}
\item This is Section-B. \ansint{0}
\end{enumerate}

\end{document}