%1
\item An arrow is shot into air. Its range is $200\m$ and its time of flight is $5\s$. If $g=10\mpss$, then horizontal component of velocity and the maximum height will be respectively
\begin{tasks}(2)   
    \task $20\mps, ~ 62.50\m$
    \task $40\mps, ~ 31.25\m$\ans
    \task $80\mps, ~ 62.5\m$
    \task None of these
\end{tasks}

%2
\item A body is projected from the ground with a velocity $\Vec{v}=(3\hat{i}+10\hat{j})\mps$. The maximum height attained and the range of the body respectively are (Take, $g=10\mpss$)
\begin{tasks}(2)
    \task $5\m$ and $6\m$\ans
    \task $3\m$ and $10\m$
    \task $6\m$ and $5\m$
    \task $3\m$ and $5\m$
\end{tasks}

%3
\item A particle moves along a parabolic path $y=-9x^2$ in such a way that the x-component of velocity remains constant and has a value $\dfrac{1}{3}\mps$. The acceleration of the particle is
\begin{tasks}(2)
    \task $\dfrac{1}{3}\mpss$
    \task $3\mpss$
    \task $\dfrac{2}{3}\mpss$
    \task $2\mpss$\ans
\end{tasks}

%4
\item A projectile can have same range from two angles of projection with same initial speed. If $h_1$ and $h_2$ be the maximum heights, then
\begin{tasks}(2)
    \task $R=\sqrt{h_1h_2}$
    \task $R=\sqrt{2h_1h_2}$
    \task $R=2\sqrt{h_1h_2}$
    \task $R=4\sqrt{h_1h_2}$\ans
\end{tasks}

%5
\item A boy can throw a stone up to a maximum height of $10\m$. The maximum horizontal distance that the boy can throw the same stone up to will be
\begin{tasks}(2)
    \task $20\sqrt{2}\m$
    \task $10\m$
    \task $10\sqrt{2}\m$
    \task $20\m$\ans
\end{tasks}