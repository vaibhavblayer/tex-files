\item The quantity $\int_{t_1}^{t_2} \vec{V} dt$ represents :
\begin{tasks}
\task Distance travelled during $t_1$ to $t_2$
\task Displacement during $t_1$ to $t_2$\ans
\task Average acceleration during $t_1$ to $t_2$
\task None of these
\end{tasks}

\item An open lift  is coming down from the top of a building at a constant speed $v=10 m/s$. A boy standing on the lift throws a stone vertically upwards at a speed of $30 m/s$ w.r.t. himself. The time after which he will catch the stone is :
\begin{tasks}(2)
\task $4 s$
\task $6 s$\ans
\task $8 s$
\task $10 s$
\end{tasks}


\item A body is thrown up from a lift with velocity $u$ relative to lift. If its time of flight with respect to lift is $t$ then acceleration of the lift is :
\begin{tasks}(2)
\task $\dfrac{(u-gt)}{t}$ upwards
\task $\dfrac{(u-gt)}{t}$ downwards
\task $\dfrac{(2u-gt)}{t}$ upwards\ans
\task $\dfrac{(2u-gt)}{t}$ downwards
\end{tasks}



\item The motion of a body falling from rest in a resisting medium is described by the equation $\dfrac{dv}{dt}=A-Bv$, where $A$ and $B$ are constants. Then :
\begin{tasks}
\task maximum possible velocity is $\dfrac{A}{B} m/s$\ans
\task initial acceleration is $A m/s^2$\ans
\task velocity at any time $t$ is $v=\dfrac{A}{B}(1-e^{-Bt})$\ans
\task velocity at any time $t$ is $v=\dfrac{A}{B}(1-e^{-At})$
\end{tasks}

%5
\item The acceleration of gravity can be measured by projecting a body upward and measuring the time that it takes to pass two given lines in both directions (upward motion and downward motion). If the time the body takes to pass a horizontal line $A$ in both direction (from $A_1$ to $A_2$) is $T_A$, and the time to go by a second line $B$ in both directions is (from $B_1$ to $B_2$)  $T_B$, then assuming that the acceleration due to gravity to be constant, its value is : 
\begin{center}
\begin{tikzpicture}[very thick,>=Stealth]
\draw (0.5, 0)--(2.5, 0);
\draw [rounded corners=8pt](01.25, 0)--(1.25, 2.5)--(1.75, 2.5)--(1.75, 0);
\node at (1.25, 0.5)[left]{$A_2$};
\node at (1.25, 0.6){$\downarrow$};
\node at (1.25, 2.0)[left]{$B_2$};
\node at (1.75, 0.5)[right]{$A_1$};
\node at (1.75, 2.0)[right]{$B_1$};
\node at (1.75, 1.9){$\uparrow$};
\draw[|<->|][thick](2.5, 0.5) --(2.5, 2);
\end{tikzpicture}
\end{center}
\begin{tasks}(2)
\task $\dfrac{8h}{T_A^2 - T_B^2}$\ans
\task $\dfrac{8h}{T_A^2 + T_B^2}$
\task $\dfrac{8h}{T_A^2  T_B^2}$
\task $\dfrac{8hT_A  T_B}{T_A^2  T_B^2}$
\end{tasks}


\item A particle is projected vertically upwards in absence of air resistance with a velocity $u$ from a point $O$. When it returns to the point of projection :
\begin{tasks}(2)
\task its average velocity is zero\ans
\task its displacement is zero\ans
\task its average speed is $u/2$\ans
\task its average speed is $u$
\end{tasks}



\item A smooth inclined plane is inclined at an angle $\theta$ with the horizontal. A body starts from rest and slides down the inclined surface. The time taken by the body to reach the bottom is :
\begin{center}
\begin{tikzpicture}[very thick,>=Stealth, xscale=0.85, yscale=0.75]
\draw (0, 0)coordinate(a)--(3, 0)coordinate(b)--(3, 3)coordinate(c)--cycle;
\draw[|<->|][thick](-0.25, 0.1) --(2.85, 3.15)node[midway,left]{$l$};
\draw[|<->|][thick](3.5, 0) --(3.5, 3)node[midway,right]{$h$};
\pic[draw=black,angle radius=8 mm, "$\theta$"]{angle=b--a--c};
\draw (2.65, 0)--(2.65, 0.35)--(3, 0.35);
\end{tikzpicture}
\end{center}
\begin{tasks}(2)
\task $\sqrt{\dfrac{2h}{g}}$
\task $\sqrt{\dfrac{2l}{g}}$
\task $\dfrac{1}{\sin\theta}\sqrt{\dfrac{2h}{g}}$\ans
\task $\sqrt{\dfrac{2l}{g\sin\theta}}$\ans
\end{tasks}

