\item The position of a particle moving along the $x$-axis is given by $x = \dfrac{t^3}{3} -2t^2 + 3t + 25$. The time(s) when particle comes to an instantaneous stop is/are :
\begin{multicols}{4}
\begin{enumerate}
\item $t=1 s$
\item $t=2 s$
\item $t=3 s$
\item $t=4 s$
\end{enumerate}
\end{multicols}

\item A particle is moving along $X-$axis whose position is given by $x=4=9t + \dfrac{t^3}{3}$. Mark the correct statement(s) in relation to its motion :
\begin{enumerate}
\item Direction of motion is not changing at any of the instants
\item Direction of motion is changing at $t=3s$
\item for $0 < t < 3 s$, the particle is slowing down
\item for $0 < t <3 s$, the particle is speeding up
\end{enumerate}


\item The position of a particle moving along $x$-axis is given by $x=3t^2-t^3$; where $x$ is in $m$ and $t$ is in $sec$. Consider the following statements:
\begin{enumerate}
\item Displacement of the particle after $4s$ is $16 m$
\item Distance travelled by the particle upto $4s$ is $24 m$
\item Displacement of the particle after $4s$ is $(-16m)$
\item Distance travelled by the particle upto $4s$ is $22 m$
\end{enumerate}

\item  A particle moves along $x$-axis such that its position varies with time as $x=50t-5t^2$. Select the correct alternative(s).
\begin{enumerate}
\item The particle has travelled $130 m$ distance by $t=6s$
\item At $t=4$ and $ t= 12 s$, the particle is at $120m$ distance from starting point.
\item The particle will never be at distance of $130m$ from starting point.
\item During the whole motion, the particle doesn't have same velocity at 2 different instants.
\end{enumerate}

\item The motion of a body falling from rest in a resisting medium is described by the equation $\dfrac{dv}{dt} = a -bv$ where $a$ and $b$ are constants. The velocity at any time $t$ is :
\begin{multicols}{4}
\begin{enumerate}
\item $v_t=\dfrac{a}{b}\left(1-e^{-bt}\right)$
\item $v_t=\dfrac{b}{a}e^{-bt}$
\item $v_t=\dfrac{a}{b}\left(1+e^{-bt}\right)$
\item $v_t=\dfrac{b}{a}e^{bt}$
\end{enumerate}
\end{multicols}

\item A particle is in motion such that its velocity parallel to $y$-axis is constant and parallel to $x$-axis it is proportion to $y$ coordinate.
\begin{enumerate}
\item The particle moves with constant acceleration
\item The path of the particle is a hyperbola
\item If the particle starts from the origin, it will initially move in $y$ direction
\item If the particle starts from the origin, it will initially move in $x$ direction
\end{enumerate}



\item A stone is dropped from the top of a tall cliff and $n$ seconds later another stone is thrown vertically downloads with a velocity $u$. Then the second stone overtakes the first, below the top of the cliff at a distance given by :
\begin{multicols}{4}
\begin{enumerate}
\item $\dfrac{g}{2}\left[\dfrac{n(gn/2-u)}{(gn-u)}\right]^2$
\item $\dfrac{g}{2}\left[\dfrac{n(gn-u/2)}{(gn-u)}\right]^2$
\item $g\left[\dfrac{(gn-u)}{(gn-u/2)}\right]^2$
\item None of these
\end{enumerate}
\end{multicols}