\documentclass{article}
\usepackage[utf8]{inputenc}
\usepackage{geometry}
\geometry{a4paper, top=15mm, bottom=20mm, left=20mm, right=20mm}
\usepackage{tzplot}
\usepackage{amsmath}
\usepackage[utopia]{mathdesign}
\usepackage{kinematikz}
\usepackage{tasks}
\newcommand{\ans}{\textcolor{red!95}{\textit{\quad Ans.}}}
\title{Module-Test-5\\(Physics-NEET)}

%\usepackage[italicdiff]{physics}
\usepackage{physunits}
\tikzstyle test-paper=[>=latex, thick]
\tikzset{
>=latex
}
\tikzstyle{block}=[rectangle,draw, thick, minimum size=8mm, node distance=1.8cm]
\tikzstyle{pulley}=[circle,draw, thick, minimum size=10mm, node distance=1.8cm]
\begin{document}
\maketitle

%	SYLLABUS
%		Section-A
% 1. Kinematics - 1->10
% 2. Projectile motion - 11->25
% 3. Laws of motion - 26->35
% 4. Mixed(Section-B) - 1->15


\begin{center}
\textbf{Section-A}
\end{center}


\begin{enumerate}

\item In the arrangement shown in figure, the ratio of tensions in the strings attached with $4 \kg$ block and that with $1 \kg$ block is
\begin{center}
\begin{tikzpicture}
\tzcoor(0, 0)(O)
	\pic[yshift=10mm,rotate=180] (hinge){frame=2.5cm};
	\node[pulley] (pulley1) at (O){};
	\node[block] (box1) [below of=pulley1, xshift=-5mm] {$4\kg$};
	\node[block] (box2) [below of=pulley1, xshift=5mm, yshift=-5mm] {$3\kg$};
	\node[block] (box3) [below of=box2]{$1\kg$};
	\tzline(box2.south)(box3.north)
	\tzdot*(pulley1.center)
	\tzline(hinge-center)(pulley1.center)
	\tzline(pulley1.west)(box1.north)
	\tzline(pulley1.east)(box2.north)
\end{tikzpicture}
\end{center} 
\begin{tasks}(2)
	\task $2:1$
	\task $4:1$\ans
	\task $1:2$
	\task $1:4$
\end{tasks}

	\item A particle is moving with a velocity $v=t^2-1$, then the distance covered from $0$ to $2\s$ is
	\begin{center}
	\begin{tikzpicture}
	\def\fn{(\x)^2-1}
		\tzfn"curve"{\fn}[0:2.1]
		\tzfnarea*[pattern=north east lines]{\fn}[0:2]
		\tzfnarealine{curve}{0}{2}
		\tzaxes(-1, -1.5)(5, 4){$t$}{$v$}
		\tzticks{1,2}{0,3}
		\tzline[->](0, 0)(5, 0)
	\end{tikzpicture}
	\end{center}
	\begin{tasks}(2)
		\task $2\m$\ans
		\task $\dfrac{2}{3}\m$
		\task $\dfrac{4}{3}\m$
		\task $\dfrac{-2}{3}\m$
	\end{tasks}
	
	\item Two particles are moving towards each other with equal speed of $5\mps$, then their relative displacement in $0$ to $5\s$ is
	\begin{tasks}(2)
		\task $50\m$\ans
		\task $25\m$
		\task $0\m$
		\task None of these
	\end{tasks}
	
	\item For the given figure, force on the block in the component form is
	\begin{center}
	\begin{tikzpicture}
		\pic[xshift=1cm](0, 0){frame=8cm};
		\draw(0, 0)rectangle(0.75, 0.75)coordinate(O) node[midway]{$m$};
		\tzcoor<1, 1>(O)(A){$20\N$}[45]
		\tzcoor<1, 0>(O)(B)
		\tzline[->](O)(A)
		\tzline[dashed](O)(B)
		\tzanglemark(B)(O)(A){$45^\circ$}(15pt)
	\end{tikzpicture}
	\end{center}
	\begin{tasks}(2)
		\task $\dfrac{10}{\sqrt{2}}\hat{i}+\dfrac{10}{\sqrt{2}}\hat{j}$
		\task $20\sqrt{2}\hat{i}+20\sqrt{2}\hat{j}$
		\task $\dfrac{20}{\sqrt{2}}\hat{i}+\dfrac{20}{\sqrt{2}}\hat{j}$\ans
		\task $10\sqrt{2}\hat{i}+10\sqrt{2}\hat{j}$\ans
	\end{tasks}
	
	\item For the given figure, acceleration of the block of mass $m$ is
	\begin{center}
	\begin{tikzpicture}
		\pic[xshift=0cm](0, 0){frame=8cm};
		\draw[xshift=-2cm](0, 0)rectangle(0.75, 0.75)coordinate(O') node[midway]{$m$};
		\draw(0, 0)rectangle(0.75, 0.75)coordinate(O) node[midway]{$m$};
		\tzcoor*<0, -0.75*0.5>(O')(O'M)
		\tzcoor*<-0.75, -0.75*0.5>(O)(OM)
		\tzline(O'M)(OM)
		\tzcoor*<0, -0.74*0.5>(O)(OMR)
		\tzline+[->](OMR)(1, 0){$F=32m$}[r]
	\end{tikzpicture}
	\end{center}
	\begin{tasks}(2)
		\task $4$
		\task $8$
		\task $16$\ans
		\task $32$
	\end{tasks}
	
\item What is the angle between $(\vec{P}+\vec{Q})$ and $(\vec{P}\times\vec{Q})$ ?
\begin{tasks}(2)
	\task zero
	\task $\dfrac{\pi}{2}$\ans
    \task $\dfrac{\pi}{4}$
	\task $\pi$
\end{tasks}

\item The displacement (in metre) of a particle moving along x-axis is given by $x=18t+5t^2$. The average acceleration during the interval $t_1=2 s$ and $t_2=4 s$ is
    \begin{tasks}(2)
            \task $13 m/s^2$
            \task $10 m/s^2$ \ans
            \task $27 m/s^2$
            \task $37 m/s^2$
    \end{tasks}

	
	\item For the given figure, tension in the string connecting the blocks of masses $m$ is
	\begin{center}
	\begin{tikzpicture}
		\pic[xshift=0cm](0, 0){frame=8cm};
		\draw[xshift=-2cm](0, 0)rectangle(0.75, 0.75)coordinate(O') node[midway]{$m$};
		\draw(0, 0)rectangle(0.75, 0.75)coordinate(O) node[midway]{$m$};
		\tzcoor*<0, -0.75*0.5>(O')(O'M)
		\tzcoor*<-0.75, -0.75*0.5>(O)(OM)
		\tzline(O'M)(OM)
		\tzcoor*<0, -0.74*0.5>(O)(OMR)
		\tzline+[->](OMR)(1, 0){$F=32m$}[r]
	\end{tikzpicture}
	\end{center}
	\begin{tasks}(2)
		\task $4m$
		\task $8m$
		\task $16m$\ans
		\task $32m$
	\end{tasks}

\item A body is projected with a velocity $20\mps$ at an angle of  $30^\circ$, then the time of flight of this projectile is
\begin{center}
\begin{tikzpicture}
	\pic[xshift=2.5cm, ] (0, 0) {frame=8cm};
	\tzparabola"curve"(0, 0)(2.5, 1.5)(5, 0);
	\tzvXpointat*{curve}{4.99}
	\tzvXpointat*{curve}{0.01}(PT)
	%\tztangentat[->]{curve}{4.99}[4.5:5.5]{$v_0$}[br]
	\tztangentat[->]{curve}{0.01}[0:0.7]{$20\mps$}
	\tzanglemark[->](5, 0)(0, 0)(PT){$30^\circ$}[scale=0.8](15pt)
\end{tikzpicture}
\end{center}
\begin{tasks}(2)
    \task $4\s$
    \task $2\s$\ans
    \task $3\s$
    \task $1\s$
\end{tasks}

\item Component of the vector $\vec{A}=2\hat{i}+3\hat{j}$ along the vector $\vec{B}=(\hat{i}+\hat{j})$ is
    \begin{tasks}(2)
            \task $\dfrac{2}{\sqrt{2}}$
            \task $4\sqrt{2}$
            \task $\dfrac{\sqrt{2}}{3}$
            \task None of these\ans
    \end{tasks}

	\item For the given figure, the acceleration of the block along the inclined plane is
	\begin{center}
	\begin{tikzpicture}
		\pic[xshift=1cm] (0, 0) {frame=8cm};
		\tzcoors(0, 0)(A)(3, 0)(B)(0,3)(C);
		\tzpolygon(A)(B)(C);
		\tzanglemark(C)(B)(A){$60^\circ$}(15pt)
		\begin{scope}[rotate around={-45:(B)}]
			\draw(0, 0)rectangle(0.5, 0.5)node[midway]{$A$};
			\tzline+[->](0.5, 0.25)(1, 0){$a$}[r]
		\end{scope}
	\end{tikzpicture}
	\end{center}
	\begin{tasks}(2)
		\task $g\cos60^\circ$
		\task $g\sin60^\circ$\ans
		\task $g\tan60^\circ$
		\task None of these
	\end{tasks}
	
	\item Three forces $\vec{F}_1$, $\vec{F}_2$ and $\vec{F}_3$ are acting on a particle of mass $m$ and vector sum of all these forces is zero. Then the acceleration of the particle is
	\begin{center}
	\begin{tikzpicture}
		\tzdot*(0, 0){$m$}[r](5pt)
		\tzline[->](0, 0)(1, 1){$\vec{F}_1$}[a]
		\tzline[->](0, 0)(-1, 1){$\vec{F}_2$}[al]
		\tzline[->](0, 0)(-1, -1){$\vec{F}_3$}[b]
	\end{tikzpicture}
	\end{center}
	\begin{tasks}(2)
		\task zero\ans
		\task Non-zero
		\task Can't say anything
		\task None of these
	\end{tasks}
	
	\item For the given figure, acceleration of the block of mass $m$ is
	\begin{center}
	\begin{tikzpicture}
		\pic[xshift=-0.5cm](0, 0){frame=8cm};
		\draw[xshift=-2cm](0, 0)rectangle(0.75, 0.75)coordinate(O') node[midway]{$m$};
		\draw(0, 0)rectangle(0.75, 0.75)coordinate(O) node[midway]{$m$};
		\tzcoor*<0, -0.75*0.5>(O')(O'M)
		\tzcoor*<-0.75, -0.75*0.5>(O)(OM)
		\tzline(O'M)(OM)
		\tzcoor*<0, -0.74*0.5>(O)(OMR)
		\tzline+[->](OMR)(1, 0){$F=32m$}[r]
		\tzcoor*<-0.75, -0.74*0.5>(O')(O'MR)
		\tzline+[->](O'MR)(-1, 0){$F=8m$}[l]
	\end{tikzpicture}
	\end{center}
	\begin{tasks}(2)
		\task $4$
		\task $8$
		\task $12$\ans
		\task $16$
	\end{tasks}
	
\item The speed of boat is $5\kmph$ in still water. It crosses a river of width $1\km$ along the shortest possible path in $15\min$. Then, velocity of river will be
\begin{tasks}(2)
    \task $4.5\kmph$
    \task $4\kmph$
    \task $1.5\kmph$
    \task $3\kmph$\ans
\end{tasks}	



\item Two blocks of masses $4 \kg$ and $2 \kg$ are attached by an inextensible light string as shown in figure. Both the blocks are pulled vertically upwards by a force $F = 120 \N$. Then, the tension in the string connecting the blocks is
\begin{center}
\begin{tikzpicture}
\tzcoor(0, 0)(O)
	\node[block] (box1) at (O) {$4\kg$};
	\node[block] (box2) [above of=box1] {$2\kg$};
	\tzline(box1.north)(box2.south)
	\tzline+[->](box2.north)(0, 1){$F=120\N$}[a]
\end{tikzpicture}
\end{center} 
\begin{tasks}(2)
	\task $60\N$
	\task $80\N$\ans
	\task $100\N$
	\task $120\N$
\end{tasks}

\item Two blocks of masses $4 \kg$ and $2 \kg$ are placed side by side on a smooth horizontal surface as shown in the figure. A horizontal force of $20 \N$ is applied on $4 \kg$ block. Then, the normal reaction between them is
\begin{center}
\begin{tikzpicture}
\tzcoor(0, 0)(O)
	\pic[yshift=-4.3mm] at (O) {frame=6cm};
	\node[block] (box1) at (O) {$4\kg$};
	\node[block] (box2) [right of=box1, xshift=-9.7mm] {$2\kg$};
	\tzline+[<-](box1.west)(-1.5, 0){$20\N$}[midway, a]
\end{tikzpicture}
\end{center} 
\begin{tasks}(2)
	\task $\dfrac{20}{3}\N$\ans
	\task $\dfrac{30}{3}\N$
	\task $\dfrac{10}{3}\N$
	\task $\dfrac{40}{3}\N$
\end{tasks}

\item Three blocks of masses $3 \kg$, $2 \kg$ and $1 \kg$ are placed side by side on a smooth surface as shown in figure. A horizontal force of $12 \N$ is applied on $3 \kg$ block. The net force on $2 \kg$ block is
\begin{center}
\begin{tikzpicture}
\tzcoor(0, 0)(O)
	\pic[yshift=-4.3mm, xshift=5mm] at (O) {frame=7cm};
	\node[block] (box1) at (O) {$3\kg$};
	\node[block] (box2) [right of=box1, xshift=-9.7mm] {$2\kg$};
	\node[block] (box3) [right of=box2, xshift=-9.7mm] {$1\kg$};
	\tzline+[<-](box1.west)(-1.5, 0){$12\N$}[midway, a]
\end{tikzpicture}
\end{center} 
\begin{tasks}(2)
	\task $2\N$
	\task $3\N$
	\task $4\N$\ans
	\task $5\N$
\end{tasks}

\item For the given $v-t$ graph, the acceleration at $t=2s$ is
    \begin{center}
        \begin{tikzpicture}
            \tzaxes[->](-1, -1)(5, 3){$t$}{$v$}
            \tztos"curve"(0, 0)[out=0, in=180]
                         (2, 2)[out=0, in=180]
                         (4, 1);
            \tzvXpointat{curve}{2}(A)
            \tzslopeat[->]{curve}{2}{1.5cm}
            \tzprojx*[dashed](A){$2$}
        \end{tikzpicture}
    \end{center}
    \begin{tasks}(2)
            \task $1 m/s^2$
            \task $-1 m/s^2$
            \task $0 m/s^2$ \ans
            \task none of these
    \end{tasks}

\item Two unequal masses of $1 \kg$ and $2 \kg$ are connected by an inextensible light string passing over a smooth pulley as shown in figure. A force $F = 20 \N$ is applied on $1 \kg$ block. The acceleration of the either block is 
\begin{center}
\begin{tikzpicture}
\tzcoor(0, 0)(O)
	\pic[yshift=10mm,rotate=180] (hinge){frame=2cm};
	\node[pulley] (pulley1) at (O){};
	\node[block] (box1) [below of=pulley1, xshift=-5mm] {$1\kg$};
	\node[block] (box2) [below of=pulley1, xshift=5mm, yshift=-5mm] {$2\kg$};
	\tzdot*(pulley1.center)
	\tzline(hinge-center)(pulley1.center)
	\tzline(pulley1.west)(box1.north)
	\tzline(pulley1.east)(box2.north)
	\tzline+[->](box1.south)(0, -1){$F$}[b]
\end{tikzpicture}
\end{center} 
\begin{tasks}(2)
	\task $\dfrac{10}{3}\mpss$\ans
	\task $\dfrac{20}{3}\mpss$
	\task $\dfrac{30}{3}\mpss$
	\task $\dfrac{40}{3}\mpss$
\end{tasks}


\item Two trains are moving with velocities $v_1=10\mps$ and $v_2=20\mps$ on the same track in opposite directions. After the application of brakes if their retarding rates are $a_1=2\mpss$ and $a_2=1\mpss$ respectively, then the minimum distance of separation between the trains to avoid collision is
\begin{tasks}(2)
    \task $150\m$
    \task $225\m$\ans
    \task $450\m$
    \task $300\m$
\end{tasks}

\item A body is projected with a velocity $20\mps$ at an angle of  $45^\circ$, then the range of this projectile is
\begin{center}
\begin{tikzpicture}
	\pic[xshift=2.5cm, ] (0, 0) {frame=8cm};
	\tzparabola"curve"(0, 0)(2.5, 1.5)(5, 0);
	\tzvXpointat*{curve}{4.99}
	\tzvXpointat*{curve}{0.01}(PT)
	%\tztangentat[->]{curve}{4.99}[4.5:5.5]{$v_0$}[br]
	\tztangentat[->]{curve}{0.01}[0:0.7]{$20\mps$}
	\tzline[|<->|]<0, -0.5>(0, 0)(5, 0){$R$}[midway, b]
	\tzanglemark[->](5, 0)(0, 0)(PT){$45^\circ$}[scale=0.8](15pt)
\end{tikzpicture}
\end{center}
\begin{tasks}(2)
    \task $40\m$\ans
    \task $20\m$
    \task $80\m$
    \task $20\sqrt{2}\m$
\end{tasks}

\item Two blocks of masses $2 \kg$ and $4 \kg$ are released from rest over a smooth inclined plane of
inclination $30^\circ$ as shown in figure. What is the normal force between the two blocks?
\begin{center}
\begin{tikzpicture}
	\pic[xshift=1cm] (0, 0) {frame=8cm};
	\tzcoors(0, 0)(A)(3, 0)(B)(0,3)(C);
	\tzpolygon(A)(B)(C);
	\tzanglemark(C)(B)(A){$30^\circ$}(15pt)
	\node[block, yshift=4mm, rotate around={-45:(B)}] (box1) at (O) {$2\kg$};
	\node[block, rotate=-45] (box2) [right of=box1, xshift=-9.7mm]{$4\kg$};
\end{tikzpicture}
\end{center}
\begin{tasks}(2)
	\task $10\N$
	\task $20\N$
	\task $5\N$
	\task Zero\ans
\end{tasks}

%23
\item Which of the following is the correct pair for the acceleration of either blocks and the tension in the string shown in figure. The pulley and the string are light and all surfaces are smooth.
\begin{center}
\begin{tikzpicture}
	\fill[pattern=north east lines](0, 0)--(8, 0)--(8, -3)--(7.75, -3)--(7.75, -0.25)--(0, -0.25)--cycle;
	\draw[thick](0, 0)--(8, 0)--(8, -3);
	\node[pulley] (pulley1) at (2, 0.9){};
	\tzline+(pulley1.center)(0, -0.9)
	\tzdot*(pulley1.center)
	\node[block] (block1) at (4.5, 0.4){$M$};
	\tzline(pulley1.south)(block1.west)
	
	\node[pulley] (pulley2) at (8+0.9*cos{45}, 0.9){};
	\tzline(pulley2.center)(8, 0)
	\tzdot*(pulley2.center)
	\node[block] (block2) at (8+0.9*cos{45}+0.5, -2){$M$};
	\tzline(pulley2.east)(block2.north)
	\tzline(pulley1.north)(pulley2.north)
\end{tikzpicture}
\end{center}
\begin{tasks}(2)
	\task $\dfrac{g}{2}, \dfrac{Mg}{2}$\ans
	\task $\dfrac{g}{3}, \dfrac{Mg}{3}$
	\task $\dfrac{g}{2}, \dfrac{Mg}{3}$
	\task $\dfrac{g}{3}, \dfrac{Mg}{2}$
\end{tasks}

\item What would be the normal reaction between the block of mass $2\kg$ and $1\kg$ in the given figure?
\begin{center}
\begin{tikzpicture}
	\pic {frame=5cm};
	\node[block] (block1) at (0, 0.4){$2\kg$};
	\node[block] (block2) [above of=block1, yshift=-1cm]{$1\kg$};
\end{tikzpicture}
\end{center}
\begin{tasks}(2)
	\task $20\N$
	\task $30\N$
	\task $10\N$\ans
	\task None of these
\end{tasks}

\item A particle moving with velocity $\vec{v}$ is acted by the resultant of three forces shown by the vector triangle $PQR$. The velocity of the particle will 
\begin{center}
\begin{tikzpicture}[scale=0.75]
\tzcoors(0, 0)(R){$R$}[b](3, 0)(Q){$Q$}[r](0, 4)(P){$P$}[a];
	\tzline[->](Q)(R)
	\tzline[->](R)(P)
	\tzline[->](P)(Q)
\end{tikzpicture}
\end{center}
\begin{tasks}(2)
	\task decrease
	\task remain constant\ans
	\task change according to smallest force $QR$
	\task increase
\end{tasks}

\item An object of mass $3\kg$ is at rest. If a force $\vec{F}=(6t^2\hat{i}+4t\hat{j})\N$ is applied on the object, then the velocity of the object at $t=3\s$ is
\begin{tasks}(2)
	\task $18\hat{i}+3\hat{j}$
	\task $18\hat{i}+6\hat{j}$\ans
	\task $3\hat{i}+18\hat{j}$
	\task $18\hat{i}+4\hat{j}$
\end{tasks}

\item An object is thrown horizontally from a tower $H$ meter high with a velocity of $\sqrt{2gH} \mps$. Its speed on striking the ground will be :
\begin{center}
	\begin{tikzpicture}
		\tzfn"curve"{-.15*(\x)^2+2}[0:4]
		\tzline[->](0, 2)(1, 2){$\sqrt{2gH}$}[r]
		\pic[xshift=3cm] (0, 0) {frame=7cm};
		\draw[pattern=north east lines](-0.1, 0 ) rectangle (0.1, 2);
		\tzline[|<->|](-0.5, 0)(-0.5, 2){$H$}[midway, left]
		\tzhXpointat*{curve}{0}(PG)
		\tzvXpointat*{curve}{0}(PT)
		%\tzline[|<->|]<0, -0.5>(0, 0)(PG){$R$}[midway, below]
	\end{tikzpicture}
	\end{center}
\begin{tasks}(2)
\task $\sqrt{2gH}$
\task $\sqrt{6gH}$
\task $2\sqrt{gH}$\ans
\task $2\sqrt{2gH}$
\end{tasks}

\item A body, under the action of a force $\vec{F}=6\hat{i}-8\hat{j}+10\hat{k}$, acquires an acceleration of $1\mpss$. The mass of this body must be
\begin{tasks}(2)
	\task $2\sqrt{10}\kg$
	\task $10\kg$
	\task $20\kg$
	\task $10\sqrt{2}\kg$\ans
\end{tasks}

\item Two bodies of mass $4\kg$ and $6\kg$ are tied to the ends of a massless string. The string passes over a pulley which is frictionless(see figure). The acceleration of the system in terms of acceleration due to gravity $g$ is
\begin{center}
\begin{tikzpicture}
	\pic[yshift=1cm, rotate=180](hinge) {frame=2cm};
	\node[pulley] (pulley1) at (0, 0){};
	\tzline(hinge-center)(pulley1.center)
	\tzdot*(pulley1.center)
	\node[block] (block1)[below of=pulley1, xshift=-5mm]{$4\kg$};
	\node[block] (block2)[below of=pulley1, xshift=5mm, yshift=-10mm]{$6\kg$};
	\tzline(pulley1.west)(block1.north)
	\tzline(pulley1.east)(block2.north)
\end{tikzpicture}
\end{center}
\begin{tasks}(2)
	\task $\dfrac{g}{2}$
	\task $\dfrac{g}{5}$\ans
	\task $\dfrac{g}{10}$
	\task $g$
\end{tasks}


\item Displacement-time equation of a particle moving along x-axis is 
    \[ x=20+t^3-12t \quad \text{(SI units)} \]
    velocity at $t=0 s$ is
    \begin{tasks}(2)
        \task $-12 m/s$ \ans
        \task $0 m/s$
        \task $+12 m/s$
        \task none of these
    \end{tasks}

\item The variation of velocity of a particle with time moving along a straight line is illustrated in the adjoining figure. The distance travelled by the particle in $2s$ is
    \begin{center}
        \begin{tikzpicture}
            \tzaxes[->](0, 0)(5, 3){$t$}{$v$}
            \tzlines*(0, 0)(1, 2)(2, 2);
            \tzproj[dashed](1, 2){$1$}{$2$}
            \tzprojx*[dashed](2, 2){$2$}
        \end{tikzpicture}
    \end{center}
    \begin{tasks}(2)
            \task $2 m$
            \task $3 m$ \ans
            \task $5 m$
            \task None of these
    \end{tasks}

    \item The motion of a particle is described by the equation $v=\alpha t$. $\alpha$ is a $+ve$ constant. The distance travelled by the particle in the first $4 s$ is
    \begin{tasks}(2)
            \task $4\alpha$
            \task $12\alpha$
            \task $6\alpha$
            \task $8\alpha$ \ans
    \end{tasks}


\item A body starting from rest has an acceleration of $4 m/s^2$. Calculate distance travelled by it in $5^{\text{th}}$ second.
    \begin{tasks}(2)
            \task $18 m$\ans
            \task $16 m$
            \task $14 m$
            \task $12 m$
    \end{tasks}

\item A particle is moving such that $s=t^3-6t^2+18t$, where $s$ is in metre and $t$ is in second. The minimum velocity attained by the particle is
\begin{center}
        \begin{tikzpicture}
        \def\Fx{0.06*((\x)^3-6*(\x)^2+18*\x)}
            \tzaxes[->](-1, -1)(5, 3){$t$}{$s$}
            \tzfn"curve"\Fx[0:4.25]
            \tztangentat[->]{curve}{3}[2:4]{$v$}
            \tzvXpointat*{curve}{3}
            %\tzfnarea*[pattern=north east lines]{\Fx}[1:2]
            %\tzfnarealine{curve}{3}

        \end{tikzpicture}
    \end{center}
    \begin{tasks}(2)
            \task $29 m/s$
            \task $5 m/s$
            \task $6 m/s$ \ans
            \task $12 m/s$
    \end{tasks}

    \item A car starts from rest, attains a velocity of $8 m/s$ with an acceleration of $4 m/s^2$, then it travels $16 m$ with this uniform velocity and then comes to rest with a uniform deceleration of $4 m/s^2$. Calculate the total time of travel of the car.
    \begin{tasks}(2)
         \task $4 s$
         \task $8 s$
         \task $12 s$
         \task None of these\ans
    \end{tasks}

\end{enumerate}


\pagebreak

\begin{center}
\textbf{Section-B}
\end{center}


\begin{enumerate}
    \item For the given $v-t$ graph, the acceleration at $t=2s$ is
    \begin{center}
        \begin{tikzpicture}
            \tzaxes[->](-1, -1)(5, 3){$t$}{$v$}
            \tztos"curve"(0, 0)[out=0, in=250]
                         (3, 2.5);
            \node at (3, 2.5)[right]{$v=\dfrac{t^2}{2}$};
            \tzvXpointat{curve}{2}(A)
            \tzslopeat[->]{curve}{2}{2cm}
            \tzprojx*[dashed](A){$2$}
        \end{tikzpicture}
    \end{center}
    \begin{tasks}(2)
            \task $4 m/s^2$
            \task $-4 m/s^2$
            \task $-2 m/s^2$
            \task $2 m/s^2$ \ans
    \end{tasks}

\item A particle is projected with a velocity of $50\mps$ at $37^\circ$ with horizontal. Find the horizontal velocity at $t=2\s$.
\begin{center}
    \begin{tikzpicture}
        \tzaxes(0, 0)(5, 3){$x$}{$y$}
        \tzline[->](0, 0)(2, 1.5){$50\mps$}[r]
        \draw (0.4, 0) arc (0:37:0.4);
    \end{tikzpicture}
\end{center}
\begin{tasks}(2)
    \task $30\mps$
    \task $40\mps$\ans
    \task $50\mps$
    \task none of these
\end{tasks}




\item Two trains are moving with velocities $v_1=10\mps$ and $v_2=20\mps$ on the same track in opposite directions. After the application of brakes if their retarding rates are $a_1=2\mpss$ and $a_2=1\mpss$ respectively, then the minimum distance of separation between the trains to avoid collision is
\begin{tasks}(2)
    \task $150\m$
    \task $225\m$\ans
    \task $450\m$
    \task $300\m$
\end{tasks}


\item A girl is walking on a horizontal road with a speed of $3\mps$. Raindrops are falling vertically downward with speed of $4\mps$ w.r.t. ground. In which direction the girl should hold her umbrella to keep the rain away ?
\begin{tasks}(2)
    \task $\dfrac{3}{5}\hat{i}+\dfrac{4}{5}\hat{j}$\ans
    \task $\dfrac{5}{3}\hat{i}+\dfrac{5}{4}\hat{j}$
    \task $\dfrac{4}{5}\hat{i}+\dfrac{3}{5}\hat{j}$
    \task none of these
\end{tasks}


\item In the figure shown, time of flight and range is
	\begin{center}
	\begin{tikzpicture}
		\tzfn"curve"{-.15*(\x)^2+2}[0:4]
		\tzline[->](0, 2)(1, 2){$20\mps$}[r]
		\pic[xshift=3cm] (0, 0) {frame=7cm};
		\draw[pattern=north east lines](-0.1, 0 ) rectangle (0.1, 2);
		\tzline[|<->|](-0.5, 0)(-0.5, 2){$20\m$}[midway, left]
		\tzhXpointat*{curve}{0}(PG)
		\tzvXpointat*{curve}{0}(PT)
		\tzline[|<->|]<0, -0.5>(0, 0)(PG){$R$}[midway, below]
	\end{tikzpicture}
	\end{center}
	\begin{tasks}(2)
		\task $2\s \text{ and } 40\m$\ans
		\task $1\s \text{ and } 20\m$
		\task $3\s \text{ and } 60\m$
		\task None of these
	\end{tasks}
	
	
\item A particle moves along the sides $AB$, $BC$, $CD$ of a square of side $25 \m$ with a velocity of $15 \mps$. Its average velocity is
	\begin{center}
	\begin{tikzpicture}
		\tzpath[draw, -->--=0.18, -->--=0.50, -->--=0.84 ](0, 0)(0, 2)(2, 2){$v$}[r](2, 0);
		\tzpath[draw, dashed](0, 0)(2, 0);
		\tznode(0, 0){$A$}[l]
		\tznode(2, 0){$D$}[r]
		\tznode(0, 2){$B$}[al]
		\tznode(2, 2){$C$}[ar]
	\end{tikzpicture}
	\end{center}
	\begin{tasks}(2)
		\task $5\mps$\ans
		\task $7.5\mps$
		\task $10\mps$
		\task $15\mps$
	\end{tasks}


\item A projectile is given an initial velocity of $(\hat{i} + 2\hat{j}) \mps$ where, $\hat{i}$ is along the ground and $\hat{j}$ is along the vertical. If $g = 10 \mpss$ , the equation of its trajectory is
	\begin{center}
	\begin{tikzpicture}
		\tzfn"curve"{-.25*(\x)^2+1*\x}[0:4]
		\pic[xshift=2cm] (0, 0) {frame=7cm};
		\tztangentat[->]{curve}{0.01}[0:1]{$\hat{i}+2\hat{j}$}[ar]
		\tzvXpointat{curve}{0.01}(PG)
		\tzanglemark[->](1, 0)(0, 0)(PG){$\theta$}(15pt)
	\end{tikzpicture}
	\end{center}
	\begin{tasks}(2)
		\task $y=x-5x^2$
		\task $y=2x-5x^2$\ans
		\task $4y=2x-5x^2$
		\task $4y=2x-25x^2$
	\end{tasks}

\item A body is projected with an angle $\theta$. The maximum height reached is $h$. If the time of flight is $4\s$ and $g=10\mpss$, then value of $h$ is
\begin{center}
\begin{tikzpicture}
	\pic[xshift=2.5cm, ] (0, 0) {frame=8cm};
	\tzparabola"curve"(0, 0)(2.5, 1.5)(5, 0);
	\tzvXpointat*{curve}{4.99}
	\tzvXpointat*{curve}{0.01}(PT)
	\tztangentat[->]{curve}{4.99}[4.5:5.5]{$v_0$}[br]
	\tztangentat[->]{curve}{0.01}[0:0.7]{$v_0$}
	\tzline[<->](2.5, 0)(2.5, 1.5){$h$}[midway, right, scale=0.8]
	\tzanglemark[->](5, 0)(0, 0)(PT){$\theta$}[scale=0.8]
\end{tikzpicture}
\end{center}
\vspace*{-30mm}
\begin{tasks}(2)
    \task $40\m$
    \task $20\m$\ans
    \task $5\m$
    \task $10\m$
\end{tasks}




\item A body is projected from the ground with a velocity $\Vec{v}=(3\hat{i}+10\hat{j})\mps$. The maximum height attained and the range of the body respectively are (Take, $g=10\mpss$)
\begin{center}
\begin{tikzpicture}
	\pic[xshift=2.5cm, ] (0, 0) {frame=8cm};
	\tzparabola"curve"(0, 0)(2.5, 1.5)(5, 0);
	\tzvXpointat*{curve}{4.99}
	\tzvXpointat*{curve}{0.01}(PT)
	%\tztangentat[->]{curve}{4.99}[4.5:5.5]{$v_0$}[br]
	\tztangentat[->]{curve}{0.01}[0:0.75]{$3\hat{i}+10\hat{j}$}
	\tzline[<->](2.5, 0)(2.5, 1.5){$H$}[midway, right, scale=0.8]
	\tzline[|<->|]<0, -0.5>(0, 0)(5, 0){$R$}[midway, b]
	\tzanglemark[->](5, 0)(0, 0)(PT){$\theta$}[scale=0.8](15pt)
\end{tikzpicture}
\end{center}
\begin{tasks}(2)
    \task $5\m$ and $6\m$\ans
    \task $3\m$ and $10\m$
    \task $6\m$ and $5\m$
    \task $3\m$ and $5\m$
\end{tasks}






\item For the given $v-t$ graph, the displacement between $t=1s$ to $t=2s$ is
    \begin{center}
        \begin{tikzpicture}
        \def\Fx{0.5*(\x)^2}
            \tzaxes[->](-1, -1)(5, 4){$t$}{$v$}
            \tzfn"curve"\Fx[0:2.5]
            \tzfnarea*[pattern=north east lines]{\Fx}[1:2]
            \node at (3, 3)[right]{$v=t^2$};
            \tzfnarealine{curve}{1}{2}
            \node at (1, 0)[below]{$1$};
            \node at (2, 0)[below]{$2$};
        \end{tikzpicture}
    \end{center}
    \begin{tasks}(2)
            \task $\dfrac{9}{3} m$
            \task $\dfrac{7}{3} m$\ans
            \task $\dfrac{8}{3} m$
            \task $\dfrac{1}{3} m$
    \end{tasks}


\item This whole system of blocks is under gravity free space, then the normal reaction between the blocks is
\begin{center}
\begin{tikzpicture}
	\pic {frame=5cm};
	\node[block] (block1) at (0, 0.4){$2\kg$};
	\node[block] (block2) [above of=block1, yshift=-1cm]{$1\kg$};
\end{tikzpicture}
\end{center}
\begin{tasks}(2)
	\task zero\ans
	\task $10\N$
	\task $20\N$
	\task $30\N$
\end{tasks}

\item In the following system of blocks and pulley, the acceleration of the block of mass $2\kg$ is
\begin{center}
\begin{tikzpicture}
	\fill[pattern=north east lines](0, 0)--(8, 0)--(8, -3)--(7.75, -3)--(7.75, -0.25)--(0, -0.25)--cycle;
	\draw[thick](0, 0)--(8, 0)--(8, -3);
	\node[pulley] (pulley1) at (1.5, 0.9){};
	\tzline+(pulley1.center)(0, -0.9)
	\tzdot*(pulley1.center)
	\node[block] (block1) at (4, 0.4){$1\kg$};
	\tzline(pulley1.south)(block1.west)
	\node[block] (block3) [right of=block1]{$3\kg$};
	\tzline(block1.east)(block3.west);
	
	\node[pulley] (pulley2) at (8+0.9*cos{45}, 0.9){};
	\tzline(pulley2.center)(8, 0)
	\tzdot*(pulley2.center)
	\node[block] (block2) at (8+0.9*cos{45}+0.5, -2){$2\kg$};
	\tzline(pulley2.east)(block2.north)
	\tzline(pulley1.north)(pulley2.north)
\end{tikzpicture}
\end{center}
\begin{tasks}(2)
	\task $\dfrac{10}{6}\mpss$
	\task $\dfrac{20}{6}\mpss$\ans
	\task $\dfrac{30}{6}\mpss$
	\task $\dfrac{40}{6}\mpss$
\end{tasks}

\item A mass of $1\kg$ is suspended by a thread. It is lifted up with an acceleration of $5\mpss$ and then it is lowered down with an acceleration of $5\mpss$. Then the ratio of tensions in the string for the both cases is
\begin{center}
\begin{tikzpicture}
\node[block] (block1) at (0, 0){$1\kg$};
\tzline+[->](block1.north)(0, 1)
\end{tikzpicture}
\end{center}
\begin{tasks}(2)
	\task $3:1$\ans
	\task $1:3$
	\task $1:2$
	\task $2:1$
\end{tasks}

\item A batsman is batting from the center in  a circular cricket ground of radius $R$. At what angle he should hit the ball so that the ball goes for the maximum distance ?
\begin{tasks}(2)
	\task $0^\circ$
	\task $30^\circ$
	\task $45^\circ$\ans
	\task $60^\circ$
\end{tasks} 

\item In the above problem at what angle he should hit the ball so that the ball goes for maximum height ?
\begin{tasks}(2)
	\task $45^\circ$
	\task $60^\circ$
	\task $75^\circ$
	\task $90^\circ$\ans
\end{tasks}


\end{enumerate}


\end{document}
