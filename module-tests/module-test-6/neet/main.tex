\documentclass{article}
\usepackage[utf8]{inputenc}
\usepackage{geometry}
\geometry{a4paper, top=15mm, bottom=20mm, left=20mm, right=20mm}
\usepackage{tzplot}
\usepackage{amsmath}
\usepackage[utopia]{mathdesign}
\usepackage{kinematikz}
\usepackage{tasks}
\newcommand{\ans}{\textcolor{red!95}{\textit{\quad Ans.}}}
\title{Module-Test-6\\(Physics-NEET)}

%\usepackage[italicdiff]{physics}
\usepackage{physunits}
\tikzstyle test-paper=[>=latex, thick]
\tikzset{
>=latex
}
\tikzstyle{block}=[rectangle,draw, thick, minimum size=8mm, node distance=1.8cm]
\tikzstyle{pulley}=[circle,draw, thick, minimum size=10mm, node distance=1.8cm]

\tikzstyle{sblock}=[rectangle,draw, thick, minimum size=8mm, node distance=1.8cm]
\tikzstyle{spulley}=[circle,draw, thick, minimum size=8mm, node distance=1.8cm]

\tikzstyle{lblock}=[rectangle,draw, thick, minimum size=12mm, node distance=1.8cm]
\tikzstyle{lpulley}=[circle,draw, thick, minimum size=12mm, node distance=2cm]

\tikzstyle{hblock}=[rectangle,draw, thick, minimum height=12mm, minimum width=20mm, node distance=1.8cm]
\tikzstyle{Hblock}=[rectangle,draw, thick, minimum height=12mm, minimum width=24mm, node distance=1.8cm]
\tikzstyle{lift}=[rectangle,draw, thick, minimum height=60mm, minimum width=50mm, node distance=1.8cm]

\tikzstyle{Bpulley}=[circle,draw, thick, minimum size=20mm, node distance=1.8cm]

\tikzstyle{plank}=[rectangle,draw, thick, minimum height=8mm, minimum width=50mm, node distance=1.8cm]

% 1. Laws of motion (1 to 21)
% 2. Projectile Motion (22 to 27)
% 3. Kinematics (28 to 30) 

\begin{document}
\maketitle

\pagebreak

\begin{enumerate}
\item[33.] In the arrangement shown in figure, the ratio of tensions in the strings attached with $4 \kg$ block and that with $1 \kg$ block is
\begin{center}
\begin{tikzpicture}
\tzcoor(0, 0)(O)
	\pic[yshift=10mm,rotate=180] (hinge){frame=2.5cm};
	\node[pulley] (pulley1) at (O){};
	\node[block] (box1) [below of=pulley1, xshift=-5mm] {$4\kg$};
	\node[block] (box2) [below of=pulley1, xshift=5mm, yshift=-5mm] {$3\kg$};
	\node[block] (box3) [below of=box2]{$1\kg$};
	\tzline(box2.south)(box3.north)
	\tzdot*(pulley1.center)
	\tzline(hinge-center)(pulley1.center)
	\tzline(pulley1.west)(box1.north)
	\tzline(pulley1.east)(box2.north)
\end{tikzpicture}
\end{center} 
\begin{tasks}(2)
	\task $2:1$
	\task $4:1$\ans
	\task $1:2$
	\task $1:4$
\end{tasks}

\item[34.] Problems of non-intertial frames can be solved only with the concept of pseudo force.
\begin{tasks}(1)
	\task Above statement is wrong\ans
	\task Above statement is right
	\task Can't say anything
	\task Above statement is right for some cases and wrong for some cases
\end{tasks}

\item[35.] A particle is dropped from a height $h$. Another particle which is initially at a horizontal distance $d$ from the first is simultaneously projected with a horizontal velocity u and the two particles just collide on the ground. Then   
\begin{tasks}(2)
	\task $d^2=\dfrac{u^2h}{2h}$
	\task $d^2=\dfrac{2u^2h}{g}$\ans
	\task $d=h$
	\task $gd^2=u^2h$
\end{tasks}

\end{enumerate}

\begin{center}
\textbf{Section-B}
\end{center}

\begin{enumerate}
\item The acceleration of the $2 \kg$ block, if the free end of string is
pulled with a force of $20 \N$ as shown, is
\begin{center}
\begin{tikzpicture}
	\pic[yshift=10mm,rotate=180] (topsurface){frame=2.5cm};
	\node[Bpulley, yshift=-20mm] (pulley1) at (topsurface-center){};
	\node[pulley, xshift=-5mm] (pulley2) [below of=pulley1]{};
	\tzline(pulley1.west)(pulley2.west)
	\tzline(pulley1.center)(pulley2.east)
	\tzline(pulley1.center)(topsurface-center)
	\tzdot*(pulley1.center)
	\tzdot*(pulley2.center)
	\tzline+[->](pulley1.east)(0, -2){$F=20\N$}[b]
	\node[block] (block1) [below of=pulley2]{$2\kg$};
	\tzline(pulley2.center)(block1.north)
\end{tikzpicture}
\end{center}
\begin{tasks}(2)
	\task zero
	\task $10\mpss$ upward\ans
	\task $5\mpss$ upward
	\task $5\mpss$ downward
\end{tasks}

\item Two bodies of masses $m_1$ and $m_2$ are connected by a light string which passes over a frictionless massless pulley. If
the pulley is moving upward with uniform acceleration $g/2$
then tension in the string will be
\begin{tasks}(2)
	\task $\dfrac{3m_1m_2}{m_1+m_2}g$\ans
	\task $\dfrac{m_1+m_2}{4m_1m_2}g$
	\task $\dfrac{2m_1m_2}{m_1+m_2}g$
	\task $\dfrac{m_1m_2}{m_1+m_2}g$
\end{tasks}


\item Consider the situation shown in figure. Both the pulleys $2\kg$ and the string are light and all the surfaces are smooth. The acceleration of $1 \kg$ block is
\begin{center}
\begin{tikzpicture}
\def\ph{0.6}%pulley-height
	\fill[pattern=north east lines](0, 0)--(8, 0)--(8, -3)--(7.75, -3)--(7.75, -0.25)--(0, -0.25)--cycle;
	\draw[thick](0, 0)--(8, 0)--(8, -3);
	\node[pulley] (pulley1) at (4, \ph){};
	\node[pulley] (pulley2) at (8+\ph*cos{45}, \ph){};
	\tzdot*(pulley1.center)
	\node[block, scale=1.25] (block1) at (1, 0.5){$2\kg$};
	\tzline(pulley1.center)(block1.10)
	\tzline(pulley1.south)(pulley2.south)
	
	\tzline(pulley2.center)(8, 0)
	\tzdot*(pulley2.center)
	\node[block, scale=1.25] (block2) at (8+\ph*cos{45}+0.5, -2){$1\kg$};
	\tzline(pulley2.east)(block2.north)
	\tzline(pulley1.north)(pulley2.north)
\end{tikzpicture}
\end{center}
\begin{tasks}(2)
	\task $\dfrac{g}{3}\mpss$
	\task $\dfrac{2g}{3}\mpss$\ans
	\task $\dfrac{4g}{3}\mpss$
	\task $\dfrac{6g}{3}\mpss$
\end{tasks}


\item Find the relation between $a_1$, $a_2$ and $a_3$ where $a_1$, $a_2$ and $a_3$ are accelerations of the blocks $1$, $2$ and $3$.
\begin{center}
\begin{tikzpicture}
\pic[yshift=1cm, rotate=180](hinge) {frame=2cm};
	\node[lpulley] (pulley1) at (0, 0){};
	\node[pulley] (pulley2)[below of=pulley1, xshift=6mm]{};
	\node[block] (block1)[below of=pulley1, xshift=-6mm]{$1$};
	\node[block] (block2)[below of=pulley2, xshift=-5mm]{$2$};
	\node[block] (block3)[below of=pulley2, xshift=5mm]{$3$};
	\tzline(hinge-center)(pulley1.center)
	\tzdot*(pulley1.center)
	\tzdot*(pulley2.center)
	
	\tzline(pulley1.west)(block1.north)
	\tzline(pulley1.east)(pulley2.center)
	\tzline(pulley2.west)(block2.north)
	\tzline(pulley2.east)(block3.north)
\end{tikzpicture}
\end{center}
\begin{tasks}(2)
	\task $2a_1+a_2+a_3=0$\ans
	\task $a_1+2a_2+a_3=0$
	\task $a_1+a_2+2a_3=0$
	\task $a_1+a_2+a_3=0$
\end{tasks}

\item In the figure shown, $a_3=6 \mpss$ (downwards) and $a_2=4\mpss$ (upwards). The acceleration of 1 is
\begin{center}
\begin{tikzpicture}
\pic[yshift=1cm, rotate=180](hinge) {frame=2cm};
	\node[lpulley] (pulley1) at (0, 0){};
	\node[pulley] (pulley2)[below of=pulley1, xshift=6mm]{};
	\node[block] (block1)[below of=pulley1, xshift=-6mm]{$1$};
	\node[block] (block2)[below of=pulley2, xshift=-5mm]{$2$};
	\node[block] (block3)[below of=pulley2, xshift=5mm]{$3$};
	\tzline(hinge-center)(pulley1.center)
	\tzdot*(pulley1.center)
	\tzdot*(pulley2.center)
	
	\tzline(pulley1.west)(block1.north)
	\tzline(pulley1.east)(pulley2.center)
	\tzline(pulley2.west)(block2.north)
	\tzline(pulley2.east)(block3.north)
\end{tikzpicture}
\end{center}
\begin{tasks}(2)
	\task $1\mpss \text{ downward}$
	\task $2\mpss \text{ upward}$
	\task $1\mpss \text{ upward}$\ans
	\task $2\mpss \text{ downward}$
\end{tasks}










\item Three blocks of masses $3 \kg$, $2 \kg$ and $1 \kg$ are placed side by side on a smooth surface as shown in figure. A horizontal force of $12 \N$ is applied on $3 \kg$ block. The net force on $2 \kg$ block is
\begin{center}
\begin{tikzpicture}
\tzcoor(0, 0)(O)
	\pic[yshift=-4.3mm, xshift=5mm] at (O) {frame=7cm};
	\node[block] (box1) at (O) {$3\kg$};
	\node[block] (box2) [right of=box1, xshift=-9.7mm] {$2\kg$};
	\node[block] (box3) [right of=box2, xshift=-9.7mm] {$1\kg$};
	\tzline+[<-](box1.west)(-1.5, 0){$12\N$}[midway, a]
\end{tikzpicture}
\end{center} 
\begin{tasks}(2)
	\task $2\N$
	\task $3\N$
	\task $4\N$\ans
	\task $5\N$
\end{tasks}

\item The surface is frictionless, the ratio between $T_1$ and $T_2$ is
\begin{center}
\begin{tikzpicture}
\pic (surface) {frame=9cm};
	\node[block, yshift=4mm] (block1) at (surface-center){$12\kg$};
	\node[block] (block2) [right of=block1]{$15\kg$};
	\node[block] (block3) [left of=block1]{$3\kg$};
	\tzline(block3.east)(block1.west){$T_1$}[midway, a]
	\tzline(block1.east)(block2.west){$T_2$}[midway, a]
	\tzline[->]"force"(block2.east)([turn]30:2)
	\tzvXpointat{force}{3}(A)
	%\tzdot*(A)
	\tzline+[dashed]"dashedline"(block2.east)(2, 0)
	\tzvXpointat{dashedline}{3}(B)
	%\tzdot*(B)
	\tzanglemark(B)(block2.east)(A){$30^\circ$}(16pt)
\end{tikzpicture}
\end{center}	
\begin{tasks}(2)
	\task $\sqrt{3}:1$
	\task $1:\sqrt{3}$
	\task $1:5$\ans
	\task $5:1$
\end{tasks}



	

\item Find the relation between $a_1$ and $a_2$ .
\begin{center}
\begin{tikzpicture}
	\pic[yshift=10mm,rotate=180] (topsurface){frame=4cm};
	\node[pulley, yshift=-15mm] (pulley1) at (topsurface-center){};
	\node[pulley, xshift=5mm] (pulley2) [below of=pulley1] {};
	\node[pulley, xshift=5mm] (pulley3) [below of=pulley2] {};
	\node[block, xshift=5mm] (block1) [below of=pulley3] {$2$};
	\tzline(topsurface-center)(pulley1.center)
	\tzdots*(pulley1.center) (pulley2.center) (pulley3.center);
	\tzline(pulley1.east)(pulley2.center)
	\tzline(pulley2.east)(pulley3.center)
	\tzline(pulley3.east)(block1.north)
	\node[plank, yshift=-50mm] (plank1)[below of=pulley1]{$1$};
	\tzline+(pulley1.west)(0, -5 - 1.4)
	\tzline+(pulley2.west)(0, -5 + 0.4)
	\tzline+(pulley3.west)(0, -5 + 2.2)
\end{tikzpicture}
\end{center}
\begin{tasks}(2)
	\task $a_1+7a_2=0$
	\task $7a_1+a_2=0$\ans
	\task $3a_1+a_2=0$
	\task $a_1+3a_2=0$
\end{tasks}


	
\item Find the force exerted by $5 \kg$ block on floor of lift, as shown in figure. (Take, $g =10\mpss$)	
\begin{center}
\begin{tikzpicture}
\node[lift] (lift) at (0, 0){};
\node[Hblock, yshift=6 mm] (block1) at (lift.south){$5\kg$};
\node[hblock, yshift=6 mm] (block2) at (block1.north){$2\kg$};
\tzline+[->](3.5, -2.5)(0, 3){$5\mpss$}[a]
\end{tikzpicture}
\end{center}
\begin{tasks}(2)
	\task $100\N$
	\task $115\N$
	\task $105\N$\ans
	\task $135\N$
\end{tasks}
	
\item In the pulley-block arrangement shown in figure, find relation between $a_A$ , $a_B$ and $a_C$ .
\begin{center}
\begin{tikzpicture}
\tzcoor(0, 0)(O)
	\pic[yshift=10mm,rotate=180] (hinge){frame=3cm};
	\node[pulley] (pulley1) at (O){};
	\node[block] (box1) [below of=pulley1, xshift=-5mm] {$A$};
	\node[pulley, xshift=5mm, yshift=-10mm] (pulley2) [below of=pulley1]{};
	\node[block] (box2) [below of=pulley2, xshift=5mm] {$C$};
	\node[block] (box3) [below of=pulley2, xshift=-5mm, yshift=-5mm]{$B$};
	\tzline(pulley2.west)(box3.north)
	\tzline(pulley2.east)(box2.north)
	\tzdot*(pulley1.center)
	\tzline(hinge-center)(pulley1.center)
	\tzline(pulley1.west)(box1.north)
	\tzline(pulley1.east)(pulley2.center)
	\tzdot*(pulley2.center)
\end{tikzpicture}
\end{center} 
\begin{tasks}(2)
	\task $2a_A+a_B+a_C=0$\ans
	\task $a_A+a_B+a_C=0$
	\task $a_A+2a_B+a_C=0$
	\task $a_A+a_B+2a_C=0$
\end{tasks}
	
\item The acceleration of particle varies with time as shown. Then the expression of $v$ as a function of time $t$ is
\begin{center}
\begin{tikzpicture}[scale=0.7]
	\tzaxes(-1, -2.5)(4.5, 3.5){$t(\s)$}{$a(\mpss)$} 
	\tzLFn(0,-2)(1, 0)[0:2.5]
	\tzticks{1}{-2}
\end{tikzpicture}
\end{center}
\begin{tasks}(2)
	\task $v=t^2-2t$\ans
	\task $v=t^2+2t$
	\task $v=-t^2+2t$
	\task $v=-t^2-2t$
\end{tasks}
	


\item A projectile is projected with speed u at an angle of $60^\circ$ with horizontal from the foot of an inclined plane. If the projectile hits the inclined plane horizontally, the range on inclined plane will be
\begin{tasks}(2)
	\task $\dfrac{u^2\sqrt{21}}{2g}$
	\task $\dfrac{3u^2}{4g}$
	\task $\dfrac{u^2}{2g}$
	\task $\dfrac{u^2\sqrt{21}}{8g}$\ans
\end{tasks}
            
            
\item A particle starts from the origin of coordinates at time $t = 0$ and moves in the $xy$ plane with a constant acceleration $\alpha$ in the $y$-direction. Its equation of motion is $y = \beta x^2$. Its velocity
component in the $x$-direction is  
\begin{tasks}(2)
	\task variable
	\task $\sqrt{\dfrac{2\alpha}{\beta}}$
	\task $\dfrac{\alpha}{2\beta}$
	\task $\sqrt{\dfrac{\alpha}{2\beta}}$\ans
\end{tasks}          
         

            
            
\item At a height of $15 \m$ from ground velocity of a projectile is $\vec{v} = (10 \hat{i} + 10\hat{j})$. Here, $\hat{j}$ is vertically upwards and $\hat{i}$ is along horizontal direction then ($g = 10 \mpss$)
\begin{tasks}(1)
	\task particle was projected at an angle of $45^\circ$ with horizontal
	\task time of flight of projectile is $4 \s$\ans
	\task horizontal range of projectile is $100 \m$
	\task maximum height of projectile from ground is $20 \m$\ans
\end{tasks}


\item The velocity of a particle depends on time $t$ as $v=t-1$, finds the displacement covered by the particle during $t=1$ to $t=3$ seconds.
    \begin{center}
        \begin{tikzpicture}[cap=round, scale=0.6]
            %\tzaxes[->](-1, -1)(5, 5)
            \draw[->] (-1, 0)--(5, 0)node[right]{$t$};
            \draw[->] (0, -2)--(0, 4)node[left]{$v$};
            \draw (0, -1)--(4, 3);
            \node at (1, 0) [below]{$1$};
            \node at (3, 0) [below]{$3$};
            \draw[dashed] (3, 0)--(3, 2);
            \draw[dashed] (0, 2)node[left]{$2$}--(3, 2);
        \end{tikzpicture}
    \end{center}

\begin{tasks}(2)
	\task $0$
    \task $2 m$\ans
    \task $-2 m$
    \task None of these
\end{tasks}

\begin{center}
\begin{tikzpicture}
\tzline(0, 0)(8, 0)
\tzline<0, -0.05>(0, 0)(8, 0)
\end{tikzpicture}
\end{center}

            
	
\end{enumerate}


\end{document}
