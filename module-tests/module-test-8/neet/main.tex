\documentclass{article}
\usepackage[utf8]{inputenc}
\usepackage{geometry}
\geometry{a4paper, top=15mm, bottom=20mm, left=20mm, right=20mm}
\usepackage{tzplot}
\usepackage{amsmath}
\usepackage[utopia]{mathdesign}
\usepackage{kinematikz}
\usepackage{tasks}
%\newcommand{\ans}{\textcolor{red!95}{\textit{\quad}}}
\newcommand{\ans}{\textcolor{red!95}{\textit{\quad Ans.}}}

\title{Module-Test-8\\(Physics-NEET)}

%\usepackage[italicdiff]{physics}
\usepackage{physunits}
\tikzstyle test-paper=[>=latex, thick]
\tikzset{
>=latex
}
\tikzstyle{block}=[rectangle,draw, thick, minimum size=8mm, node distance=1.8cm]
\tikzstyle{pulley}=[circle,draw, thick, minimum size=10mm, node distance=1.8cm]

\tikzstyle{sblock}=[rectangle,draw, thick, minimum size=8mm, node distance=1.8cm]
\tikzstyle{spulley}=[circle,draw, thick, minimum size=8mm, node distance=1.8cm]

\tikzstyle{lblock}=[rectangle,draw, thick, minimum size=12mm, node distance=1.8cm]
\tikzstyle{lpulley}=[circle,draw, thick, minimum size=12mm, node distance=2cm]

\tikzstyle{hblock}=[rectangle,draw, thick, minimum height=12mm, minimum width=20mm, node distance=1.8cm]
\tikzstyle{Hblock}=[rectangle,draw, thick, minimum height=12mm, minimum width=24mm, node distance=1.8cm]
\tikzstyle{lift}=[rectangle,draw, thick, minimum height=60mm, minimum width=50mm, node distance=1.8cm]

\tikzstyle{Bpulley}=[circle,draw, thick, minimum size=20mm, node distance=1.8cm]

\tikzstyle{plank}=[rectangle,draw, thick, minimum height=8mm, minimum width=50mm, node distance=1.8cm]

% 1. Laws of motion (1 to 21)
% 2. Projectile Motion (22 to 27)
% 3. Kinematics (28 to 30) 

\begin{document}
\maketitle

\begin{enumerate}

\item A heavy uniform chain lies on horizontal table top. If the coefficient of friction between the chain and the table surface is $0.25$, then the maximum friction of the length of the chain that can hang over the edge of the table is
\begin{tasks}(2)
	\task $20\%$\ans
	\task $25\%$
	\task $35\%$
	\task $15\%$
\end{tasks}

\item A block $B$ is pushed momentarily along a horizontal surface with an initial velocity $v$. If $\mu$ is the coefficient of sliding friction between $B$ and the surface, block $B$ will come to rest after a time 
\begin{center}
\begin{tikzpicture}
\pic {frame=5cm};
\node[block, yshift=4mm] (box1) at (0, 0) {$B$};
\tzline+[->](box1.east)(1.5, 0){$v$}[r]
\end{tikzpicture}
\end{center}
\begin{tasks}(2)
	\task $\dfrac{v}{\mu g}$\ans
	\task $\dfrac{\mu g}{v}$
	\task $\dfrac{g}{v}$
	\task $\dfrac{v}{g}$
\end{tasks}

\item The upper half of an inclined plane of inclination $\theta$ is perfectly smooth while lower half is rough. A block starting from rest at the top of the bottom, if the coefficient of friction between the block and the lower half of the plane is given by
\begin{tasks}(2)
	\task $\mu = \dfrac{1}{\tan\theta}$
	\task $\mu = \dfrac{2}{\tan\theta}$
	\task $\mu = 2\tan\theta$\ans
	\task $\mu = \tan\theta$
\end{tasks}

\item The acceleration of the $2 \kg$ block, if the free end of string is
pulled with a force of $20 \N$ as shown, is
\begin{center}
\begin{tikzpicture}
	\pic[yshift=10mm,rotate=180] (topsurface){frame=2.5cm};
	\node[Bpulley, yshift=-20mm] (pulley1) at (topsurface-center){};
	\node[pulley, xshift=-5mm] (pulley2) [below of=pulley1]{};
	\tzline(pulley1.west)(pulley2.west)
	\tzline(pulley1.center)(pulley2.east)
	\tzline(pulley1.center)(topsurface-center)
	\tzdot*(pulley1.center)
	\tzdot*(pulley2.center)
	\tzline+[->](pulley1.east)(0, -2){$F=20\N$}[b]
	\node[block] (block1) [below of=pulley2]{$2\kg$};
	\tzline(pulley2.center)(block1.north)
\end{tikzpicture}
\end{center}
\begin{tasks}(2)
	\task zero
	\task $10\mpss$ upward\ans
	\task $5\mpss$ upward
	\task $5\mpss$ downward
\end{tasks}

\item A block of mass $10\kg$ is placed on a rough horizontal surface having coefficient of friction $\mu = 0.5$. If a horizontal force of $100\N$ is applied on it, then the acceleration of the block will be
\begin{tasks}(2)
	\task $15\mpss$
	\task $10\mpss$
	\task $5\mpss$\ans
	\task $0.5\mpss$
\end{tasks}

\item Consider, a car moving along a straight horizontal road with a speed of $72\kmph$. If the coefficient of static friction between the tyres and the road is $0.5$, the shortest distance in which the car can be stopped is
\begin{tasks}(2)
	\task $30\m$
	\task $40\m$\ans
	\task $72\m$
	\task $20\m$
\end{tasks}

\item Two unequal masses of $1 \kg$ and $2 \kg$ are connected by an inextensible light string passing over a smooth pulley as shown in figure. A force $F = 20 \N$ is applied on $1 \kg$ block. The acceleration of the either block is 
\begin{center}
\begin{tikzpicture}
\tzcoor(0, 0)(O)
	\pic[yshift=10mm,rotate=180] (hinge){frame=2cm};
	\node[pulley] (pulley1) at (O){};
	\node[block] (box1) [below of=pulley1, xshift=-5mm] {$1\kg$};
	\node[block] (box2) [below of=pulley1, xshift=5mm, yshift=-5mm] {$2\kg$};
	\tzdot*(pulley1.center)
	\tzline(hinge-center)(pulley1.center)
	\tzline(pulley1.west)(box1.north)
	\tzline(pulley1.east)(box2.north)
	\tzline+[->](box1.south)(0, -1){$F$}[b]
\end{tikzpicture}
\end{center} 
\begin{tasks}(2)
	\task $\dfrac{10}{3}\mpss$\ans
	\task $\dfrac{20}{3}\mpss$
	\task $\dfrac{30}{3}\mpss$
	\task $\dfrac{40}{3}\mpss$
\end{tasks}

\item A body of mass $m$ is kept on a rough horizontal surface( coefficient of friction is $\mu$ ). Horizontal force is applied on the body, but it does not move. The resultant of normal reaction and the frictional force acting on the object is given $F$, where $F$ is
\begin{tasks}(2)
	\task $|F|=mg+\mu mg$
	\task $|F|=\mu mg$
	\task $|F|\leq mg\sqrt{1+\mu^2}$\ans
	\task $|F|=mg$
\end{tasks}

\item A block of mass $10\kg$ is kept on a rough inclined plane as shown in the figure. A force of $3 \N$ is applied on the block. The coefficient of static friction between the plane and the block is $0.6$. What should be the minimum value of force $F$ , such that the block does not move downward ?
\begin{center}
\begin{tikzpicture}
	\pic[xshift=1cm] (0, 0) {frame=8cm};
	\tzcoors(0, 0)(A)(3, 0)(B)(0,3)(C);
	\tzpolygon(A)(B)(C);
	\tzanglemark(C)(B)(A){$45^\circ$}(15pt)
	\node[block, yshift=4mm, rotate around={-45:(B)}] (box1) at (0, 0) {$10\kg$};
	\tzline+[->] (box1.east)(1*cos{45}, -1*sin{45}){$3\N$}[b]
	\tzline+[->] (box1.west)(-1*cos{45}, 1*sin{45}){$F$}[a]
\end{tikzpicture}
\end{center}
\begin{tasks}(2)
	\task $32\N$\ans
	\task $25\N$
	\task $23\N$
	\task $18\N$
\end{tasks}


\item A light string passing over a smooth light pulley connects two blocks of masses $m_1$ and $m_2$ (vertically). If the acceleration of the system is $g/8$,then the ratio of the masses is
\begin{tasks}(2)
	\task $8:1$
	\task $9:7$\ans
	\task $4:3$
	\task $5:3$
\end{tasks}

\item Three identical blocks of masses $m = 2 \kg$ are drawn by a force $F=10.2 \N$ with an acceleration of $0.6 \mpss$ on a frictionless surface, then what is the tension (in $\N$) in the string between the blocks $B$ and $C$ ?
\begin{center}
\begin{tikzpicture}
\pic (surface) {frame=9cm};
	\node[block, yshift=4mm] (block1) at (surface-center){$B$};
	\node[block] (block2) [right of=block1]{$A$};
	\node[block] (block3) [left of=block1]{$C$};
	\tzline[-->--=0.5](block3.east)(block1.west)
	\tzline[-->--=0.5](block1.east)(block2.west)
	\tzline+[->](block2.east)(1.5, 0){$F$}[r]
\end{tikzpicture}
\end{center}	
\begin{tasks}(2)
	\task $9.2$
	\task $7.8$\ans
	\task $4$
	\task $9.8$
\end{tasks}

\item When forces $F_1$ , $F_2$ , $F_3$ are acting on a particle
of mass $m$ such that $F_2$ and $F_3$ are mutually perpendicular, then the particle remains stationary. If the force $F_1$ is now removed, then the acceleration of the particle is
\begin{tasks}(2)
	\task $\dfrac{F_1}{m}$\ans
	\task $\dfrac{F_2F_3}{mF_1}$
	\task $\dfrac{F_2-F_3}{m}$
	\task $\dfrac{F_2}{m}$
\end{tasks}


\item A marble block of mass $2 \kg$ lying on ice when given a velocity of $6\mps$ is stopped by friction in $10 \s$. Then, the coefficient of friction is
\begin{tasks}(2)
	\task $0.02$
	\task $0.03$
	\task $0.06$\ans
	\task $0.01$
\end{tasks}

\item The acceleration of particle varies with time as shown. Then the expression of $v$ as a function of time $t$ is
\begin{center}
\begin{tikzpicture}[scale=0.7]
	\tzaxes(-1, -2.5)(4.5, 3.5){$t(\s)$}{$a(\mpss)$} 
	\tzLFn(0,-2)(1, 0)[0:2.5]
	\tzticks{1}{-2}
\end{tikzpicture}
\end{center}
\begin{tasks}(2)
	\task $v=t^2-2t$\ans
	\task $v=t^2+2t$
	\task $v=-t^2+2t$
	\task $v=-t^2-2t$
\end{tasks}

\item A block of mass $M$ is pulled along a horizontal frictionless surface by a rope of mass $m$. If a force $P$ is applied at the free end of the rope, the force exerted by the rope on the block is
\begin{tasks}(2)
	\task $\dfrac{Pm}{M+m}$
	\task $\dfrac{Pm}{M-m}$
	\task $P$
	\task $\dfrac{PM}{M+m}$\ans
\end{tasks}


\item Consider a car moving on a straight road with a speed of $100 \mps$. The distance at which car can be stopped, is [$\mu_k = 0.5$]
\begin{tasks}(2)
	\task $800\m$
	\task $1000\m$\ans 
	\task $100\m$
	\task $400\m$
\end{tasks}


\item In the arrangement shown in figure, the ratio of tensions in the strings attached with $4 \kg$ block and that with $1 \kg$ block is
\begin{center}
\begin{tikzpicture}
\tzcoor(0, 0)(O)
	\pic[yshift=10mm,rotate=180] (hinge){frame=2.5cm};
	\node[pulley] (pulley1) at (O){};
	\node[block] (box1) [below of=pulley1, xshift=-5mm] {$4\kg$};
	\node[block] (box2) [below of=pulley1, xshift=5mm, yshift=-5mm] {$3\kg$};
	\node[block] (box3) [below of=box2]{$1\kg$};
	\tzline(box2.south)(box3.north)
	\tzdot*(pulley1.center)
	\tzline(hinge-center)(pulley1.center)
	\tzline(pulley1.west)(box1.north)
	\tzline(pulley1.east)(box2.north)
\end{tikzpicture}
\end{center} 
\begin{tasks}(2)
	\task $2:1$
	\task $4:1$\ans
	\task $1:2$
	\task $1:4$
\end{tasks}

\item A smooth block is released at rest on a $45^\circ$ incline and then slides a distance $d$.The time taken to slide is $n$ times as much to slide on rough incline than on a smooth incline. The coefficient of friction is
\begin{tasks}(2)
	\task $\mu_k=1-\dfrac{1}{n^2}$\ans
	\task $\mu_k=\sqrt{1-\dfrac{1}{n^2}}$
	\task $\mu_s=1-\dfrac{1}{n^2}$
	\task $\mu_s=\sqrt{1-\dfrac{1}{n^2}}$
\end{tasks}

\item A mass of $10 \kg$ is suspended vertically by a rope from the roof. When a horizontal force is applied on the mass, the rope deviated at an angle of $45^\circ$ at the roof point. If the suspended mass is at equilibrium, the magnitude of the force applied is
\begin{tasks}(2)
	\task $70\N$
	\task $200\N$
	\task $100\N$\ans
	\task $140\N$
\end{tasks}


\item Two blocks of equal mass are stacked on top of each other on a horizontal plane, then the frictional force between them is
\begin{tasks}(2)
	\task $0$\ans
	\task $\infty$
	\task can't say
	\task none of these
\end{tasks}

\item A block of mass $m$ is placed on a frictionless inclined plane, then the angle of repose is 
\begin{tasks}(2)
	\task $45^\circ$
	\task $30^\circ$
	\task $0^\circ$\ans
	\task None of these
\end{tasks}

\item A block of mass $m$ is placed on a frictionless horizontal plane, then the angle of friction is 
\begin{tasks}(2)
	\task $45^\circ$
	\task $60^\circ$
	\task $30^\circ$
	\task $0^\circ$\ans
\end{tasks}

\item A uniform cube of mass $m$ and side $a$ is resting in equilibrium on a rough $45^\circ$ inclined surface. The distance of the point of application of normal reaction measured from the lower edge of the cube is 
\begin{tasks}(2)
	\task $\dfrac{a}{2}$
	\task $\dfrac{a}{4}$
	\task $\dfrac{a}{5}$
	\task $0$\ans
\end{tasks}


\item Three blocks of masses $3 \kg$, $2 \kg$ and $1 \kg$ are placed side by side on a smooth surface as shown in figure. A horizontal force of $12 \N$ is applied on $3 \kg$ block. The net force on $2 \kg$ block is
\begin{center}
\begin{tikzpicture}
\tzcoor(0, 0)(O)
	\pic[yshift=-4.3mm, xshift=5mm] at (O) {frame=7cm};
	\node[block] (box1) at (O) {$3\kg$};
	\node[block] (box2) [right of=box1, xshift=-9.7mm] {$2\kg$};
	\node[block] (box3) [right of=box2, xshift=-9.7mm] {$1\kg$};
	\tzline+[<-](box1.west)(-1.5, 0){$12\N$}[midway, a]
\end{tikzpicture}
\end{center} 
\begin{tasks}(2)
	\task $2\N$
	\task $3\N$
	\task $4\N$\ans
	\task $5\N$
\end{tasks}

\item A thin rod of length $1 \m$ is fixed in a vertical position inside a train, which is moving horizontally with constant acceleration $4 \mpss$. A bead can slide on the rod and friction
coefficient between them is $0.5$. If the bead is released from rest at the top of the rod, it will reach the bottom in time $t$ then the value of $2t$ is
\begin{tasks}(2)
	\task $1$\ans
	\task $2$
	\task $4$
	\task $0$
\end{tasks}

\item The velocity of a particle depends on time $t$ as $v=t-1$, finds the displacement covered by the particle during $t=1$ to $t=3$ seconds.
    \begin{center}
        \begin{tikzpicture}[cap=round, scale=0.6]
            %\tzaxes[->](-1, -1)(5, 5)
            \draw[->] (-1, 0)--(5, 0)node[right]{$t$};
            \draw[->] (0, -2)--(0, 4)node[left]{$v$};
            \draw (0, -1)--(4, 3);
            \node at (1, 0) [below]{$1$};
            \node at (3, 0) [below]{$3$};
            \draw[dashed] (3, 0)--(3, 2);
            \draw[dashed] (0, 2)node[left]{$2$}--(3, 2);
        \end{tikzpicture}
    \end{center}

\begin{tasks}(2)
	\task $0$
    \task $2 m$\ans
    \task $-2 m$
    \task None of these
\end{tasks}

\item The minimum force required to start pushing a body up a rough (frictional coefficient $\mu$) inclined plane is $F_1$ while the minimum force needed to prevent it from sliding down is $F_2$. If the inclined plane makes an angle $\theta$ from the horizontal such
that $\tan\theta = 2\mu$, then the ratio $\dfrac{F_1}{F_2}$ is
\begin{tasks}(2)
	\task $4$
	\task $1$
	\task $2$
	\task $3$\ans
\end{tasks}

\item A block kept on a rough inclined plane, as shown in the figure, remains at rest upto a maximum force $2 \N$ down the inclined plane. The maximum external force up the inclined plane that does not move the block is $10 \N$. The coefficient of static friction between the block and the plane is (Take, $g = 10 \mpss$)
\begin{center}
\begin{tikzpicture}
	\pic[xshift=1cm] (0, 0) {frame=8cm};
	\tzcoors(0, 0)(A)(3, 0)(B)(0,3)(C);
	\tzpolygon(A)(B)(C);
	\tzanglemark(C)(B)(A){$30^\circ$}(15pt)
	\node[block, yshift=4mm, rotate around={-45:(B)}] (box1) at (0, 0) {};
	\tzline+[->] (box1.east)(1*cos{45}, -1*sin{45}){$2\N$}[b]
	\tzline+[->, dashed] (box1.west)(-1*cos{45}, 1*sin{45}){$10\N$}[a]
\end{tikzpicture}
\end{center}
\begin{tasks}(2)
	\task $\dfrac{2}{3}$
	\task $\dfrac{\sqrt{3}}{2}$\ans
	\task $\dfrac{\sqrt{3}}{4}$
	\task $\dfrac{1}{2}$
\end{tasks}


\item A block has been placed on an inclined plane with the slope angle $\theta$, block slides down the plane at constant speed. The coefficient of kinetic friction is equal to
\begin{tasks}(2)
	\task $\sin\theta$
	\task $\cos\theta$
	\task $g$
	\task $\tan\theta$\ans
\end{tasks}

\item Three blocks of masses $3 \kg$, $2 \kg$ and $1 \kg$ are placed side by side on a smooth surface as shown in figure. A horizontal force of $12 \N$ is applied on $3 \kg$ block. The net force on $2 \kg$ block is
\begin{center}
\begin{tikzpicture}
\tzcoor(0, 0)(O)
	\pic[yshift=-4.3mm, xshift=5mm] at (O) {frame=7cm};
	\node[block] (box1) at (O) {$3\kg$};
	\node[block] (box2) [right of=box1, xshift=-9.7mm] {$2\kg$};
	\node[block] (box3) [right of=box2, xshift=-9.7mm] {$1\kg$};
	\tzline+[<-](box1.west)(-1.5, 0){$12\N$}[midway, a]
\end{tikzpicture}
\end{center} 
\begin{tasks}(2)
	\task $2\N$
	\task $3\N$
	\task $4\N$\ans
	\task $5\N$
\end{tasks}

\item A monkey of mass $20\kg$ is holding a vertical rope. The rope will not break, when a mass of $25\kg$ is suspended from it but will break, if the mass exceeds $25\kg$. What is the maximum acceleration with which the monkey can climb up along the rope ?
\begin{tasks}(2)
	\task $25\mpss$
	\task $2.5\mpss$\ans
	\task $5\mpss$
	\task $10\mpss$
\end{tasks}

\item A man weighs $80\kg$. He stands on a weighing  scale in a lift which is moving upwards with a uniform acceleration of $5\mpss$. What would be the reading on the scale?
\begin{tasks}(2)
	\task $800\N$
	\task $1200\N$\ans
	\task zero
	\task $400\N$
\end{tasks}

\item For the given figure, acceleration of the block of mass $m$ is
	\begin{center}
	\begin{tikzpicture}
		\pic[xshift=0cm](0, 0){frame=8cm};
		\draw[xshift=-2cm](0, 0)rectangle(0.75, 0.75)coordinate(O') node[midway]{$m$};
		\draw(0, 0)rectangle(0.75, 0.75)coordinate(O) node[midway]{$m$};
		\tzcoor*<0, -0.75*0.5>(O')(O'M)
		\tzcoor*<-0.75, -0.75*0.5>(O)(OM)
		\tzline(O'M)(OM)
		\tzcoor*<0, -0.74*0.5>(O)(OMR)
		\tzline+[->](OMR)(1, 0){$F=32m$}[r]
	\end{tikzpicture}
	\end{center}
	\begin{tasks}(2)
		\task $4$
		\task $8$
		\task $16$\ans
		\task $32$
	\end{tasks}

\item A lift of mass $1000\kg$ is moving upwards with an acceleration of $1\mpss$. The tension developed in the string, which is connected to lift is ($g=9.8\mpss$)
\begin{tasks}(2)
	\task $9800\N$
	\task $10800\N$\ans
	\task $11000\N$
	\task $10000\N$
\end{tasks}

\item Two blocks of masses $4 \kg$ and $2 \kg$ are placed side by side on a smooth horizontal surface as shown in the figure. A horizontal force of $20 \N$ is applied on $4 \kg$ block. Then, the normal reaction between them is
\begin{center}
\begin{tikzpicture}
\tzcoor(0, 0)(O)
	\pic[yshift=-4.3mm] at (O) {frame=6cm};
	\node[block] (box1) at (O) {$4\kg$};
	\node[block] (box2) [right of=box1, xshift=-9.7mm] {$2\kg$};
	\tzline+[<-](box1.west)(-1.5, 0){$20\N$}[midway, a]
\end{tikzpicture}
\end{center} 
\begin{tasks}(2)
	\task $\dfrac{20}{3}\N$\ans
	\task $\dfrac{30}{3}\N$
	\task $\dfrac{10}{3}\N$
	\task $\dfrac{40}{3}\N$
\end{tasks}

\end{enumerate}

\begin{center}
\textbf{Section-B}
\end{center}

\begin{enumerate}

\item Two bodies of masses $m_1$ and $m_2$ are connected by a light string which passes over a frictionless massless pulley. If
the pulley is moving upward with uniform acceleration $g/2$
then tension in the string will be
\begin{tasks}(2)
	\task $\dfrac{3m_1m_2}{m_1+m_2}g$\ans
	\task $\dfrac{m_1+m_2}{4m_1m_2}g$
	\task $\dfrac{2m_1m_2}{m_1+m_2}g$
	\task $\dfrac{m_1m_2}{m_1+m_2}g$
\end{tasks}


\item Consider the situation shown in figure. Both the pulleys $2\kg$ and the string are light and all the surfaces are smooth. The acceleration of $1 \kg$ block is
\begin{center}
\begin{tikzpicture}
\def\ph{0.6}%pulley-height
	\fill[pattern=north east lines](0, 0)--(8, 0)--(8, -3)--(7.75, -3)--(7.75, -0.25)--(0, -0.25)--cycle;
	\draw[thick](0, 0)--(8, 0)--(8, -3);
	\node[pulley] (pulley1) at (4, \ph){};
	\node[pulley] (pulley2) at (8+\ph*cos{45}, \ph){};
	\tzdot*(pulley1.center)
	\node[block, scale=1.25] (block1) at (1, 0.5){$2\kg$};
	\tzline(pulley1.center)(block1.10)
	\tzline(pulley1.south)(pulley2.south)
	
	\tzline(pulley2.center)(8, 0)
	\tzdot*(pulley2.center)
	\node[block, scale=1.25] (block2) at (8+\ph*cos{45}+0.5, -2){$1\kg$};
	\tzline(pulley2.east)(block2.north)
	\tzline(pulley1.north)(pulley2.north)
\end{tikzpicture}
\end{center}
\begin{tasks}(2)
	\task $\dfrac{g}{3}\mpss$
	\task $\dfrac{2g}{3}\mpss$\ans
	\task $\dfrac{4g}{3}\mpss$
	\task $\dfrac{6g}{3}\mpss$
\end{tasks}


\item Find the relation between $a_1$, $a_2$ and $a_3$ where $a_1$, $a_2$ and $a_3$ are accelerations of the blocks $1$, $2$ and $3$.
\begin{center}
\begin{tikzpicture}
\pic[yshift=1cm, rotate=180](hinge) {frame=2cm};
	\node[lpulley] (pulley1) at (0, 0){};
	\node[pulley] (pulley2)[below of=pulley1, xshift=6mm]{};
	\node[block] (block1)[below of=pulley1, xshift=-6mm]{$1$};
	\node[block] (block2)[below of=pulley2, xshift=-5mm]{$2$};
	\node[block] (block3)[below of=pulley2, xshift=5mm]{$3$};
	\tzline(hinge-center)(pulley1.center)
	\tzdot*(pulley1.center)
	\tzdot*(pulley2.center)
	
	\tzline(pulley1.west)(block1.north)
	\tzline(pulley1.east)(pulley2.center)
	\tzline(pulley2.west)(block2.north)
	\tzline(pulley2.east)(block3.north)
\end{tikzpicture}
\end{center}
\begin{tasks}(2)
	\task $2a_1+a_2+a_3=0$\ans
	\task $a_1+2a_2+a_3=0$
	\task $a_1+a_2+2a_3=0$
	\task $a_1+a_2+a_3=0$
\end{tasks}

\item In the figure shown, $a_3=6 \mpss$ (downwards) and $a_2=4\mpss$ (upwards). The acceleration of 1 is
\begin{center}
\begin{tikzpicture}
\pic[yshift=1cm, rotate=180](hinge) {frame=2cm};
	\node[lpulley] (pulley1) at (0, 0){};
	\node[pulley] (pulley2)[below of=pulley1, xshift=6mm]{};
	\node[block] (block1)[below of=pulley1, xshift=-6mm]{$1$};
	\node[block] (block2)[below of=pulley2, xshift=-5mm]{$2$};
	\node[block] (block3)[below of=pulley2, xshift=5mm]{$3$};
	\tzline(hinge-center)(pulley1.center)
	\tzdot*(pulley1.center)
	\tzdot*(pulley2.center)
	
	\tzline(pulley1.west)(block1.north)
	\tzline(pulley1.east)(pulley2.center)
	\tzline(pulley2.west)(block2.north)
	\tzline(pulley2.east)(block3.north)
\end{tikzpicture}
\end{center}
\begin{tasks}(2)
	\task $1\mpss \text{ downward}$
	\task $2\mpss \text{ upward}$
	\task $1\mpss \text{ upward}$\ans
	\task $2\mpss \text{ downward}$
\end{tasks}










\item Three blocks of masses $3 \kg$, $2 \kg$ and $1 \kg$ are placed side by side on a smooth surface as shown in figure. A horizontal force of $12 \N$ is applied on $3 \kg$ block. The net force on $2 \kg$ block is
\begin{center}
\begin{tikzpicture}
\tzcoor(0, 0)(O)
	\pic[yshift=-4.3mm, xshift=5mm] at (O) {frame=7cm};
	\node[block] (box1) at (O) {$3\kg$};
	\node[block] (box2) [right of=box1, xshift=-9.7mm] {$2\kg$};
	\node[block] (box3) [right of=box2, xshift=-9.7mm] {$1\kg$};
	\tzline+[<-](box1.west)(-1.5, 0){$12\N$}[midway, a]
\end{tikzpicture}
\end{center} 
\begin{tasks}(2)
	\task $2\N$
	\task $3\N$
	\task $4\N$\ans
	\task $5\N$
\end{tasks}

\item The surface is frictionless, the ratio between $T_1$ and $T_2$ is
\begin{center}
\begin{tikzpicture}
\pic (surface) {frame=9cm};
	\node[block, yshift=4mm] (block1) at (surface-center){$12\kg$};
	\node[block] (block2) [right of=block1]{$15\kg$};
	\node[block] (block3) [left of=block1]{$3\kg$};
	\tzline(block3.east)(block1.west){$T_1$}[midway, a]
	\tzline(block1.east)(block2.west){$T_2$}[midway, a]
	\tzline[->]"force"(block2.east)([turn]30:2)
	\tzvXpointat{force}{3}(A)
	%\tzdot*(A)
	\tzline+[dashed]"dashedline"(block2.east)(2, 0)
	\tzvXpointat{dashedline}{3}(B)
	%\tzdot*(B)
	\tzanglemark(B)(block2.east)(A){$30^\circ$}(16pt)
\end{tikzpicture}
\end{center}	
\begin{tasks}(2)
	\task $\sqrt{3}:1$
	\task $1:\sqrt{3}$
	\task $1:5$\ans
	\task $5:1$
\end{tasks}



\item A projectile is projected with speed u at an angle of $60^\circ$ with horizontal from the foot of an inclined plane. If the projectile hits the inclined plane horizontally, the range on inclined plane will be
\begin{tasks}(2)
	\task $\dfrac{u^2\sqrt{21}}{2g}$
	\task $\dfrac{3u^2}{4g}$
	\task $\dfrac{u^2}{2g}$
	\task $\dfrac{u^2\sqrt{21}}{8g}$\ans
\end{tasks}
            
            
\item A particle starts from the origin of coordinates at time $t = 0$ and moves in the $xy$ plane with a constant acceleration $\alpha$ in the $y$-direction. Its equation of motion is $y = \beta x^2$. Its velocity
component in the $x$-direction is  
\begin{tasks}(2)
	\task variable
	\task $\sqrt{\dfrac{2\alpha}{\beta}}$
	\task $\dfrac{\alpha}{2\beta}$
	\task $\sqrt{\dfrac{\alpha}{2\beta}}$\ans
\end{tasks}   

\item A particle is moving such that $s=t^3-6t^2+18t$, where $s$ is in metre and $t$ is in second. The minimum velocity attained by the particle is
\begin{center}
        \begin{tikzpicture}
        \def\Fx{0.06*((\x)^3-6*(\x)^2+18*\x)}
            \tzaxes[->](-1, -1)(5, 3){$t$}{$s$}
            \tzfn"curve"\Fx[0:4.25]
            \tztangentat[->]{curve}{3}[2:4]{$v$}
            \tzvXpointat*{curve}{3}
            %\tzfnarea*[pattern=north east lines]{\Fx}[1:2]
            %\tzfnarealine{curve}{3}

        \end{tikzpicture}
    \end{center}
    \begin{tasks}(2)
            \task $29 m/s$
            \task $5 m/s$
            \task $6 m/s$ \ans
            \task $12 m/s$
    \end{tasks}


\item In the figure shown, time of flight and range is
	\begin{center}
	\begin{tikzpicture}
		\tzfn"curve"{-.15*(\x)^2+2}[0:4]
		\tzline[->](0, 2)(1, 2){$20\mps$}[r]
		\pic[xshift=3cm] (0, 0) {frame=7cm};
		\draw[pattern=north east lines](-0.1, 0 ) rectangle (0.1, 2);
		\tzline[|<->|](-0.5, 0)(-0.5, 2){$20\m$}[midway, left]
		\tzhXpointat*{curve}{0}(PG)
		\tzvXpointat*{curve}{0}(PT)
		\tzline[|<->|]<0, -0.5>(0, 0)(PG){$R$}[midway, below]
	\end{tikzpicture}
	\end{center}
	\begin{tasks}(2)
		\task $2\s \text{ and } 40\m$\ans
		\task $1\s \text{ and } 20\m$
		\task $3\s \text{ and } 60\m$
		\task None of these
	\end{tasks}	

            
            
\item At a height of $15 \m$ from ground velocity of a projectile is $\vec{v} = (10 \hat{i} + 10\hat{j})$. Here, $\hat{j}$ is vertically upwards and $\hat{i}$ is along horizontal direction then ($g = 10 \mpss$)
\begin{tasks}(1)
	\task particle was projected at an angle of $45^\circ$ with horizontal
	\task time of flight of projectile is $4 \s$\ans
	\task horizontal range of projectile is $100 \m$
	\task maximum height of projectile from ground is $40 \m$
\end{tasks}

\item A mass of $1\kg$ is suspended by a thread. It is lifted up with an acceleration of $5\mpss$ and then it is lowered down with an acceleration of $5\mpss$. Then the ratio of tensions in the string for the both cases is
\begin{center}
\begin{tikzpicture}
\node[block] (block1) at (0, 0){$1\kg$};
\tzline+[->](block1.north)(0, 1)
\end{tikzpicture}
\end{center}
\begin{tasks}(2)
	\task $3:1$\ans
	\task $1:3$
	\task $1:2$
	\task $2:1$
\end{tasks}

\item A body starting from rest has an acceleration of $4 m/s^2$. Calculate distance travelled by it in $5^{\text{th}}$ second.
    \begin{tasks}(2)
            \task $18 m$\ans
            \task $16 m$
            \task $14 m$
            \task $12 m$
    \end{tasks}

	
\item Find the force exerted by $5 \kg$ block on floor of lift, as shown in figure. (Take, $g =10\mpss$)	
\begin{center}
\begin{tikzpicture}
\node[lift] (lift) at (0, 0){};
\node[Hblock, yshift=6 mm] (block1) at (lift.south){$5\kg$};
\node[hblock, yshift=6 mm] (block2) at (block1.north){$2\kg$};
\tzline+[->](3.5, -2.5)(0, 3){$5\mpss$}[a]
\end{tikzpicture}
\end{center}
\begin{tasks}(2)
	\task $100\N$
	\task $115\N$
	\task $105\N$\ans
	\task $135\N$
\end{tasks}
	
\item In the pulley-block arrangement shown in figure, find relation between $a_A$ , $a_B$ and $a_C$ .
\begin{center}
\begin{tikzpicture}
\tzcoor(0, 0)(O)
	\pic[yshift=10mm,rotate=180] (hinge){frame=3cm};
	\node[pulley] (pulley1) at (O){};
	\node[block] (box1) [below of=pulley1, xshift=-5mm] {$A$};
	\node[pulley, xshift=5mm, yshift=-10mm] (pulley2) [below of=pulley1]{};
	\node[block] (box2) [below of=pulley2, xshift=5mm] {$C$};
	\node[block] (box3) [below of=pulley2, xshift=-5mm, yshift=-5mm]{$B$};
	\tzline(pulley2.west)(box3.north)
	\tzline(pulley2.east)(box2.north)
	\tzdot*(pulley1.center)
	\tzline(hinge-center)(pulley1.center)
	\tzline(pulley1.west)(box1.north)
	\tzline(pulley1.east)(pulley2.center)
	\tzdot*(pulley2.center)
\end{tikzpicture}
\end{center} 
\begin{tasks}(2)
	\task $2a_A+a_B+a_C=0$\ans
	\task $a_A+a_B+a_C=0$
	\task $a_A+2a_B+a_C=0$
	\task $a_A+a_B+2a_C=0$
\end{tasks}
	
	
       
         






\begin{center}
\begin{tikzpicture}
\tzline(0, 0)(8, 0)
\tzline<0, -0.05>(0, 0)(8, 0)
\end{tikzpicture}
\end{center}

            
	
\end{enumerate}


\end{document}
