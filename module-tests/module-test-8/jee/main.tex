\documentclass{article}
\usepackage[utf8]{inputenc}
\usepackage{geometry}
\geometry{a4paper, top=15mm, bottom=20mm, left=20mm, right=20mm}
\usepackage{tzplot}
\usepackage{amsmath}
\usepackage[utopia]{mathdesign}
\usepackage{kinematikz}
\usepackage{tasks}
%\newcommand{\ans}{\textcolor{red!95}{\textit{\quad}}}
%\def\ansit#1{\textcolor{red!95}{\quad}}
\newcommand{\ans}{\textcolor{red!95}{\textit{\quad Ans.}}}
\def\ansit#1{\textcolor{red!95}{\quad [ #1 ]}}


\title{Module-Test-8\\(Physics-JEE)}

\usepackage{physunits}
\tikzstyle test-paper=[>=latex, thick]
\tikzset{
>=latex
}

\def\surface#1{
	\fill[pattern=north east lines](0, 0)--(8, 0)--(8, -3)--(7.75, -3)--(7.75, -0.25)--(0, -0.25)--cycle;
	\draw[thick](0, 0)--(8, 0)--(8, -3);
	}

\tikzstyle{block}=[rectangle,draw, thick, minimum size=8mm, node distance=1.8cm]
\tikzstyle{pulley}=[circle,draw, thick, minimum size=10mm, node distance=1.8cm]

\tikzstyle{sblock}=[rectangle,draw, thick, minimum size=8mm, node distance=1.8cm]
\tikzstyle{spulley}=[circle,draw, thick, minimum size=8mm, node distance=1.8cm]

\tikzstyle{lblock}=[rectangle,draw, thick, minimum size=12mm, node distance=1.8cm]
\tikzstyle{lpulley}=[circle,draw, thick, minimum size=12mm, node distance=2cm]

\tikzstyle{hblock}=[rectangle,draw, thick, minimum height=12mm, minimum width=20mm, node distance=1.8cm]
\tikzstyle{Hblock}=[rectangle,draw, thick, minimum height=12mm, minimum width=24mm, node distance=1.8cm]
\tikzstyle{lift}=[rectangle,draw, thick, minimum height=60mm, minimum width=50mm, node distance=1.8cm]

\tikzstyle{Bpulley}=[circle,draw, thick, minimum size=20mm, node distance=1.8cm]

\tikzstyle{plank}=[rectangle,draw, thick, minimum height=8mm, minimum width=50mm, node distance=1.8cm]



%friction

% block pushed against a wall -> f_min for the block to be static
%			      -> f_max for the block to be in motion
% block is placed on an inclinded plane
% 


\begin{document}
\maketitle

\begin{center}
\textbf{Section-A\\(One Options Correct Type)}

\flushleft{This section contains 20 multiple choice questions. Each question has four choices (A), (B), (C) and
  (D), out of which ONLY ONE option is correct.}
  
\rule{\textwidth}{1 pt}
\end{center}

\begin{enumerate}
\item In the figure shown, the frictional coefficient between table and block is $0.2$. Find the ratio of tensions in the right and left strings.
\begin{center}
\begin{tikzpicture}
\def\ph{1}%pulley-height
\def\tl{7}%table-length
	\fill[pattern=north east lines](0, -3)--(0, 0)--(\tl, 0)--(\tl, -3)--(\tl - 0.25, -3)--(\tl - 0.25, -0.25)--(0.25, -0.25)--(0.25, -3)--(0, -3)--cycle;
	\draw[thick](0, -3)--(0, 0)--(\tl, 0)--(\tl, -3);
	\node[pulley] (pulley1) at (-\ph, 0){};
	\node[pulley] (pulley2) at (\tl+\ph, 0){};
	\tzdot*(pulley1.center)
	\node[block, scale=1.25] (block1) at (-\ph-0.5, -2){$15\kg$};
	\node[block, scale=1.25] (block3) at (0.5*\tl, 0.5){$5\kg$};
	\tzline(pulley1.west)(block1.north)
	\tzline[-->--](block3.west)(pulley1.north)
	\tzline[-->--](block3.east)(pulley2.north)
	
	\tzline(pulley2.center)(\tl, 0)
	\tzline(pulley1.center)(0, 0)
	\tzdot*(pulley2.center)
	\node[block, scale=1.25] (block2) at (\tl+\ph+0.5, -2){$5\kg$};
	\tzline(pulley2.east)(block2.north)
\end{tikzpicture}
\end{center}
\begin{tasks}(2)
	\task $17:24$\ans
	\task $34:12$
	\task $2:3$
	\task $3:2$
\end{tasks}


\item To mop-clean a floor, a cleaning machine presses a circular mop of radius $R$ vertically down with a total force $F$ and rotates it with a constant angular speed about its axis. If the force $F$ is distributed uniformly over the mop and if coefficient of friction between the mop and the floor is $\mu$, the torque applied by the machine on the mop in $(\N\m)$ is
\begin{tasks}(2)
	\task $\dfrac{2}{3}\mu FR$\ans
	\task $\dfrac{1}{6}\mu FR$
	\task $\dfrac{1}{3}\mu FR$
	\task $\dfrac{1}{2}\mu FR$
\end{tasks}

\item A particle of mass $m$ is moving in a straight line with momentum $p$. Starting at time $t = 0$, a force $F = kt$ acts in the same direction on the moving particle during time interval $T$ , so that its momentum changes from $p$ to $3p$. Here, $k$ is a constant. The value of $T$ is
\begin{tasks}(2)
	\task $\sqrt{\dfrac{2p}{k}}$
	\task $2\sqrt{\dfrac{p}{k}}$\ans
	\task $\sqrt{\dfrac{2k}{p}}$
	\task $2\sqrt{\dfrac{k}{p}}$
\end{tasks}

\item A block kept on a rough inclined plane, as shown in the figure, remains at rest upto a maximum force $2 \N$ down the inclined plane. The maximum external force up the inclined plane that does not move the block is $10 \N$. The coefficient of static friction between the block and the plane is (Take, $g = 10 \mpss$)
\begin{center}
\begin{tikzpicture}
	\pic[xshift=1cm] (0, 0) {frame=8cm};
	\tzcoors(0, 0)(A)(3, 0)(B)(0,3)(C);
	\tzpolygon(A)(B)(C);
	\tzanglemark(C)(B)(A){$30^\circ$}(15pt)
	\node[block, yshift=4mm, rotate around={-45:(B)}] (box1) at (0, 0) {};
	\tzline+[->] (box1.east)(1*cos{45}, -1*sin{45}){$2\N$}[b]
	\tzline+[->, dashed] (box1.west)(-1*cos{45}, 1*sin{45}){$10\N$}[a]
\end{tikzpicture}
\end{center}
\begin{tasks}(2)
	\task $\dfrac{2}{3}$
	\task $\dfrac{\sqrt{3}}{2}$\ans
	\task $\dfrac{\sqrt{3}}{4}$
	\task $\dfrac{1}{2}$
\end{tasks}


\item A mass of $10 \kg$ is suspended vertically by a rope from the roof. When a horizontal force is applied on the mass, the rope deviated at an angle of $45^\circ$ at the roof point. If the suspended mass is at equilibrium, the magnitude of the force applied is
\begin{tasks}(2)
	\task $70\N$
	\task $200\N$
	\task $100\N$\ans
	\task $140\N$
\end{tasks}


\item A block of mass m is placed on a surface with
a vertical cross-section given by $y = x^3 / 6$. If
the coefficient of friction is $0.5$, the maximum height above the ground at which the block can be placed without slipping is
\begin{tasks}(2)
	\task $\dfrac{1}{6}\m$\ans
	\task $\dfrac{2}{3}\m$
	\task $\dfrac{1}{3}\m$
	\task $\dfrac{1}{2}\m$
\end{tasks}

\item The minimum force required to start pushing a body up a rough (frictional coefficient $\mu$) inclined plane is $F_1$ while the minimum force needed to prevent it from sliding down is $F_2$. If the inclined plane makes an angle $\theta$ from the horizontal such
that $\tan\theta = 2\mu$, then the ratio $\dfrac{F_1}{F_2}$ is
\begin{tasks}(2)
	\task $4$
	\task $1$
	\task $2$
	\task $3$\ans
\end{tasks}

\item A block of mass m is connected to another
block of mass $M$ by a spring (massless) of spring constant $k$ . The blocks are kept on a smooth horizontal plane. Initially the blocks are at rest and the spring is unstretched. Then, a constant force $F$ starts acting on the block of mass $M$ to pull it. Find the force on the block of mass $m$.
\begin{tasks}(2)
	\task $\dfrac{mF}{M}$
	\task $\dfrac{(M+m)F}{m}$
	\task $\dfrac{mF}{m+M}$\ans
	\task $\dfrac{MF}{m+M}$
\end{tasks}

\item A mass of $M \kg$ is suspended by a weightless
string. The horizontal force that is required
to displace it until the string makes an angle
of $45^\circ$ with the initial vertical direction is
\begin{tasks}(2)
	\task $Mg(\sqrt{2} + 1)$
	\task $Mg\sqrt{2}$
	\task $\dfrac{Mg}{\sqrt{2}}$
	\task $Mg(\sqrt{2}-1)$\ans
\end{tasks}

\item Consider a car moving on a straight road with a speed of $100 \mps$. The distance at which car can be stopped, is [$\mu_k = 0.5$]
\begin{tasks}(2)
	\task $800\m$
	\task $1000\m$\ans 
	\task $100\m$
	\task $400\m$
\end{tasks}

\item A smooth block is released at rest on a $45^\circ$ incline and then slides a distance $d$.The time taken to slide is $n$ times as much to slide on rough incline than on a smooth incline. The coefficient of friction is
\begin{tasks}(2)
	\task $\mu_k=1-\dfrac{1}{n^2}$\ans
	\task $\mu_k=\sqrt{1-\dfrac{1}{n^2}}$
	\task $\mu_s=1-\dfrac{1}{n^2}$
	\task $\mu_s=\sqrt{1-\dfrac{1}{n^2}}$
\end{tasks}

\item The upper half of an inclined plane with inclination $\theta$ is perfectly smooth, while the lower half is rough. A body starting from rest at the top will again come to rest at the bottom, if the coefficient of friction for the lower half is given by
\begin{tasks}(2)
	\task $2\sin\theta$
	\task $2\cos\theta$
	\task $2\tan\theta$\ans
	\task $\tan\theta$
\end{tasks}

\item A block rests on a rough inclined plane making an angle of $30^\circ$ with the horizontal. The coefficient of static friction between the block and the plane is $0.8$. If the frictional force on the block is $10 \N$, the mass of the block (in $\kg$) is ( $g = 10 \mpss$)
\begin{tasks}(2)
	\task $2.0$\ans
	\task $4.0$
	\task $1.6$
	\task $2.5$
\end{tasks}

\item A marble block of mass $2 \kg$ lying on ice when given a velocity of $6\mps$ is stopped by friction in $10 \s$. Then, the coefficient of friction is
\begin{tasks}(2)
	\task $0.02$
	\task $0.03$
	\task $0.06$\ans
	\task $0.01$
\end{tasks}

\item A block of mass $M$ is pulled along a horizontal frictionless surface by a rope of mass $m$. If a force $P$ is applied at the free end of the rope, the force exerted by the rope on the block is
\begin{tasks}(2)
	\task $\dfrac{Pm}{M+m}$
	\task $\dfrac{Pm}{M-m}$
	\task $P$
	\task $\dfrac{PM}{M+m}$\ans
\end{tasks}

\item A light string passing over a smooth light pulley connects two blocks of masses $m_1$ and $m_2$ (vertically). If the acceleration of the system is $g/8$,then the ratio of the masses is
\begin{tasks}(2)
	\task $8:1$
	\task $9:7$\ans
	\task $4:3$
	\task $5:3$
\end{tasks}

\item Three identical blocks of masses $m = 2 \kg$ are drawn by a force $F=10.2 \N$ with an acceleration of $0.6 \mpss$ on a frictionless surface, then what is the tension (in $\N$) in the string between the blocks $B$ and $C$ ?
\begin{center}
\begin{tikzpicture}
\pic (surface) {frame=9cm};
	\node[block, yshift=4mm] (block1) at (surface-center){$B$};
	\node[block] (block2) [right of=block1]{$A$};
	\node[block] (block3) [left of=block1]{$C$};
	\tzline[-->--=0.5](block3.east)(block1.west)
	\tzline[-->--=0.5](block1.east)(block2.west)
	\tzline+[->](block2.east)(1.5, 0){$F$}[r]
\end{tikzpicture}
\end{center}	
\begin{tasks}(2)
	\task $9.2$
	\task $7.8$\ans
	\task $4$
	\task $9.8$
\end{tasks}

\item When forces $F_1$ , $F_2$ , $F_3$ are acting on a particle
of mass $m$ such that $F_2$ and $F_3$ are mutually perpendicular, then the particle remains stationary. If the force $F_1$ is now removed, then the acceleration of the particle is
\begin{tasks}(2)
	\task $\dfrac{F_1}{m}$\ans
	\task $\dfrac{F_2F_3}{mF_1}$
	\task $\dfrac{F_2-F_3}{m}$
	\task $\dfrac{F_2}{m}$
\end{tasks}

\item A block of mass $10\kg$ is kept on a rough inclined plane as shown in the figure. A force of $3 \N$ is applied on the block. The coefficient of static friction between the plane and the block is $0.6$. What should be the minimum value of force $F$ , such that the block does not move downward ?
\begin{center}
\begin{tikzpicture}
	\pic[xshift=1cm] (0, 0) {frame=8cm};
	\tzcoors(0, 0)(A)(3, 0)(B)(0,3)(C);
	\tzpolygon(A)(B)(C);
	\tzanglemark(C)(B)(A){$45^\circ$}(15pt)
	\node[block, yshift=4mm, rotate around={-45:(B)}] (box1) at (0, 0) {$10\kg$};
	\tzline+[->] (box1.east)(1*cos{45}, -1*sin{45}){$3\N$}[b]
	\tzline+[->] (box1.west)(-1*cos{45}, 1*sin{45}){$F$}[a]
\end{tikzpicture}
\end{center}
\begin{tasks}(2)
	\task $32\N$\ans
	\task $25\N$
	\task $23\N$
	\task $18\N$
\end{tasks}

\item A block of mass m is placed at rest on a horizontal rough surface with angle of friction $\phi$. The block is pulled with a force $F$ at an angle $\theta$ with the horizontal. The minimum value of $F$ required to move the block is
\begin{tasks}(2)
	\task $\dfrac{mg\sin\phi}{\cos(\theta-\phi)}$\ans
	\task $\dfrac{mg\cos\phi}{\cos(\theta-\phi)}$
	\task $mg\tan\phi$
	\task $mg\sin\phi$
\end{tasks}


\end{enumerate}



\begin{center}
\textbf{Section-B\\(Numerical Answer Type)}

{\flushleft{This section contains 10 questions. The answer to each question is a NUMERICAL VALUE. For each question, enter the correct numerical value (in decimal notation, truncated/rounded-off to the second decimal place).}}\\

\textbf{Do any 5 questions out of 10 Questions.}

\rule{\textwidth}{1 pt}
\end{center}


\begin{enumerate}
\item A thin rod of length $1 \m$ is fixed in a vertical position inside a train, which is moving horizontally with constant acceleration $4 \mpss$. A bead can slide on the rod and friction
coefficient between them is $0.5$. If the bead is released from rest at the top of the rod, it will reach the bottom in time $t$ then the value of $2t$ is
\ansit{1}

\item A uniform cube of mass $m$ and side $a$ is resting in equilibrium on a rough $45^\circ$ inclined surface. The distance of the point of application of normal reaction measured from the lower edge of the cube is \ansit{0}

\item A block $A$ of mass $2 \kg$ rests on another block $B$ of mass $8 \kg$ which rests on a horizontal floor. The coefficient of friction between $A$ and $B$ is $0.2$ while that between $B$ and floor is $0.5$. When a horizontal force $F$ of $25 \N$ is applied on the block $B$, the force of friction between $A$ and $B$ is \ansit{0}

\item A block $A$ of mass $4 \kg$ is kept on ground. The coefficient of friction between the block and the ground is $0.8$. The external force of magnitude $30 \N$ is applied parallel to the ground. The resultant force exerted by the ground on the block in newton is ($g = 10 \mpss$ ) \ansit{50}

\item A horizontal force of $10 \N$ is necessary to just
hold a block stationary against a wall. The
coefficient of friction between the block and
the wall is $0.2$. The weight of the block is \ansit{2}

\item A block rests on a rough inclined plane making an angle of $30^\circ$ with the horizontal. The coefficient of static friction between the block and the plane is $0.8$. If the frictional force on the block is $10 \N$, the mass of the block (in $\kg$) is ( $g = 10 \mpss$)\ansit{2}

\item A spring balance is attached to the ceiling of a lift. A man hangs his bag on the spring and the spring reads $49 \N$, when the lift is stationary. If the lift moves downward with an acceleration of $5 \mpss$, the reading of the spring balance will be \ansit{24}

\item Two blocks of equal mass are stacked on top of each other on a horizontal plane, then the frictional force between them is \ansit{0}

\item A block of mass $m$ is placed on a frictionless inclined plane, then the angle of repose is \ansit{0}

\item A block of mass $m$ is placed on a frictionless horizontal plane, then the angle of friction is \ansit{0}
	
\end{enumerate}


\end{document}
